\chapter{Functions of Two Variables}
\section{Functions of Two Variables}
\subsection*{Learning Objectives}
After completing this section, you should be able to:
\begin{itemize}
  \item Understand the concept of a function of two variables.
  \item Interpret functions of two variables using formulas, tables, and graphs.
  \item Evaluate functions of two variables at given input values.
  \item Identify the domain and range of a function of two variables.
  \item Apply functions of two variables in business and economic contexts.
\end{itemize}

\subsection*{Key Definitions and Concepts}

\textbf{Function of Two Variables.}  
A function of two variables assigns a single output value to each ordered pair $(x,y)$ in its domain. We write
\[
z = f(x,y).
\]

\medskip

\textbf{Inputs and Output.}  
The variables $x$ and $y$ are the independent variables, and $z$ is the dependent variable.

\medskip

\textbf{Domain.}  
The domain of a function of two variables is the set of all ordered pairs $(x,y)$ for which the function is defined.

\medskip

\textbf{Range.}  
The range is the set of all possible output values of the function.

\medskip

\textbf{Business Interpretation.}  
In business applications, functions of two variables are often used to model cost, revenue, demand, and profit depending on two inputs such as time and quantity or price and advertising.

\subsection*{Solved Examples}

\textbf{Example 1: Cost as a Function of Two Variables}

\textbf{Problem.}  
Suppose the cost of renting a car depends on the number of days rented, $d$, and the number of miles driven, $m$.  
Explain what the notation $C(d,m)$ represents.

\textbf{Solution.}

The function $C(d,m)$ represents the total cost of renting the car when it is rented for $d$ days and driven $m$ miles.

\medskip

\textbf{Example 2: Evaluating a Function from a Formula}

\textbf{Problem.}  
Suppose the cost function is
\[
C(d,m)=40d+0.15m.
\]
Find $C(3,200)$ and interpret the result.

\textbf{Solution.}

\[
C(3,200)=40(3)+0.15(200)=120+30=150.
\]

\textbf{Interpretation.}  
The cost of renting the car for 3 days and driving 200 miles is \$150.

\medskip

\textbf{Example 3: Order Matters}

\textbf{Problem.}  
Using the same function, compare $C(100,4)$ and $C(4,100)$.

\textbf{Solution.}

\[
C(100,4)=40(100)+0.15(4)=4000.60,
\]
\[
C(4,100)=40(4)+0.15(100)=175.
\]

Since the inputs represent different quantities, the order of the inputs matters.

\medskip

\textbf{Example 4: Graphing a Function of Two Variables}

\textbf{Problem.}  
Describe the graph of the function $z=2$.

\textbf{Solution.}

The graph of $z=2$ is a plane parallel to the $xy$-plane, located 2 units above it.

\medskip

\textbf{Example 5: Distance in Three Dimensions}

\textbf{Problem.}  
Find the distance between the points $A=(1,2,3)$ and $B=(7,5,-3)$.

\textbf{Solution.}

\[
\text{Distance}
=
\sqrt{(7-1)^2+(5-2)^2+(-3-3)^2}
=
\sqrt{36+9+36}
=
9.
\]

\subsection*{Practice Problems}

\begin{enumerate}
  \item A company’s revenue depends on the number of units sold, $q$, and the price per unit, $p$.  
  Explain what the notation $R(q,p)$ represents.
  \workbox[2.0in]{}

  \item Suppose the cost function is
  \[
  C(x,y)=5x+2y.
  \]
  \begin{enumerate}
    \item Find $C(4,10)$.
    \workbox[1.8in]{}
    \item Explain the meaning of this value in context.
    \workbox[2.2in]{}
  \end{enumerate}

  \item Determine whether $C(2,5)$ and $C(5,2)$ are equal for the function
  \[
  C(x,y)=3x+4y.
  \]
  Explain why or why not.
  \workbox[2.4in]{}

  \item Describe the graph of the function $z=-1$.
  \workbox[1.6in]{}

  \item Find the distance between the points $(2,0,1)$ and $(5,4,5)$.
  \workbox[2.4in]{}

  \begin{answer}
  \textbf{Functions of Two Variables, Practice Problems.}
  \begin{enumerate}
    \item $R(q,p)$ represents the revenue generated by selling $q$ units at price $p$.
    \item
    \begin{enumerate}
      \item $C(4,10)=40.$
      \item The total cost for $x=4$ and $y=10$ is \$40.
    \end{enumerate}
    \item $C(2,5)=26$ and $C(5,2)=23$; the order of inputs matters.
    \item A plane parallel to the $xy$-plane, 1 unit below it.
    \item $\sqrt{34}.$
  \end{enumerate}
  \end{answer}
\end{enumerate}

\subsection*{Section Summary}

\begin{itemize}
  \item Functions of two variables assign one output to each ordered pair of inputs.
  \item Order matters when evaluating functions of multiple variables.
  \item Graphs of functions of two variables are surfaces in three dimensions.
  \item These functions are widely used in business and economics.
  \item Understanding functions of two variables prepares us for partial derivatives.
\end{itemize}


\newpage

\section{Calculus of Functions of Two Variables}

\subsection*{Learning Objectives}
After completing this section, you should be able to:
\begin{itemize}
  \item Compute partial derivatives of functions of two variables.
  \item Interpret partial derivatives in practical and business contexts.
  \item Estimate partial derivatives from tables and contour diagrams.
  \item Use partial derivatives to estimate function values.
  \item Construct and interpret linear (tangent plane) approximations.
\end{itemize}

\subsection*{Key Definitions and Concepts}

\textbf{Partial Derivative.}  
The partial derivative of a function $f(x,y)$ with respect to $x$ measures how $f$ changes as $x$ changes while $y$ is held constant. It is denoted by
\[
f_x(x,y).
\]
Similarly, the partial derivative with respect to $y$ is denoted by
\[
f_y(x,y).
\]

\medskip

\textbf{Interpretation.}  
Partial derivatives represent rates of change. In business applications, they often describe how a quantity such as cost, revenue, or demand responds to changes in one variable while other variables remain fixed.

\medskip

\textbf{Linear Approximation (Tangent Plane).}  
Near a point $(a,b)$, a function $f(x,y)$ can be approximated by
\[
f(x,y) \approx f(a,b)+f_x(a,b)(x-a)+f_y(a,b)(y-b).
\]
This approximation is useful for estimating values when exact computation is difficult.

\subsection*{Solved Examples}

\textbf{Example 1: Computing Partial Derivatives}

\textbf{Problem.}  
Let
\[
f(x,y)=x^2-4xy+4y^2.
\]
Find $f_x$ and $f_y$.

\textbf{Solution.}

Treat $y$ as a constant when differentiating with respect to $x$:
\[
f_x(x,y)=2x-4y.
\]

Treat $x$ as a constant when differentiating with respect to $y$:
\[
f_y(x,y)=-4x+8y.
\]

\medskip

\textbf{Example 2: Evaluating Partial Derivatives}

\textbf{Problem.}  
Using the same function, find $f_x(1,1)$ and $f_y(1,1)$.

\textbf{Solution.}

\[
f_x(1,1)=2(1)-4(1)=-2,
\qquad
f_y(1,1)=-4(1)+8(1)=4.
\]

\medskip

\textbf{Example 3: Interpretation}

\textbf{Problem.}  
If $f(x,y)$ represents revenue depending on price $x$ and advertising spending $y$, explain the meaning of $f_x$.

\textbf{Solution.}

The partial derivative $f_x$ represents the rate at which revenue changes as price changes, assuming advertising spending remains constant.

\medskip

\textbf{Example 4: Linear Approximation}

\textbf{Problem.}  
Suppose $f(1,1)=2$, $f_x(1,1)=0.5$, and $f_y(1,1)=1$.  
Estimate $f(1.1,0.9)$.

\textbf{Solution.}

\[
f(1.1,0.9)\approx 2+0.5(0.1)+1(-0.1)=2.
\]

\subsection*{Practice Problems}

\begin{enumerate}
  \item Let $f(x,y)=x^2+3y^2$.
  \begin{enumerate}
    \item Find $f_x$ and $f_y$.
    \workbox[1.6in]{}
    \item Evaluate $f_x(2,1)$ and $f_y(2,1)$.
    \workbox[1.8in]{}
  \end{enumerate}

  \item Suppose $C(x,y)$ represents cost depending on labor $x$ and materials $y$.  
  Explain the meaning of $C_y(x,y)$.
  \workbox[2.2in]{}

  \item Given
  \[
  g(x,y)=\frac{e^{x+y}}{3}+\ln(y),
  \]
  find $g_x$ and $g_y$.
  \workbox[2.2in]{}

  \item The partial derivatives of $f$ at $(2,3)$ are $f_x=1.2$ and $f_y=-0.5$.  
  Use linear approximation to estimate $f(2.1,2.9)$.
  \workbox[2.0in]{}

  \item Explain why partial derivatives are useful for estimating values of multivariable functions.
  \workbox[2.2in]{}

  \begin{answer}
  \textbf{Calculus of Functions of Two Variables, Practice Problems.}
  \begin{enumerate}
    \item
    \begin{enumerate}
      \item $f_x=2x,\; f_y=6y.$
      \item $f_x(2,1)=4,\; f_y(2,1)=6.$
    \end{enumerate}
    \item $C_y$ measures how cost changes as materials change while labor is fixed.
    \item $g_x=\frac{1}{3}e^{x+y},\; g_y=\frac{1}{3}e^{x+y}+\frac{1}{y}.$
    \item $f(2.1,2.9)\approx f(2,3)+1.2(0.1)-0.5(-0.1)=f(2,3)+0.17.$
    \item Partial derivatives approximate how outputs change in response to small input changes.
  \end{enumerate}
  \end{answer}
\end{enumerate}

\subsection*{Section Summary}

\begin{itemize}
  \item Partial derivatives measure rates of change in one variable at a time.
  \item They play a key role in business modeling and economic interpretation.
  \item Linear approximations allow estimation near known points.
  \item These ideas extend single-variable calculus to more realistic models.
\end{itemize}

\newpage

\section{Optimization}

\subsection*{Learning Objectives}
After completing this section, you should be able to:
\begin{itemize}
  \item Compute second partial derivatives of functions of two variables.
  \item Find and classify critical points.
  \item Identify local maxima, local minima, and saddle points.
  \item Apply the second derivative test for functions of two variables.
  \item Solve applied optimization problems involving business models.
\end{itemize}

\subsection*{Key Definitions and Concepts}

\textbf{Critical Point.}  
A point $(a,b)$ is a critical point of $f(x,y)$ if
\[
f_x(a,b)=0 \quad \text{and} \quad f_y(a,b)=0,
\]
or if one or both partial derivatives do not exist.

\medskip

\textbf{Second Partial Derivatives.}  
The second partial derivatives of $f(x,y)$ are
\[
f_{xx}, \quad f_{yy}, \quad f_{xy}, \quad f_{yx}.
\]
If these derivatives are continuous, then $f_{xy}=f_{yx}$.

\medskip

\textbf{Discriminant.}  
At a critical point $(a,b)$, define
\[
D(a,b)=f_{xx}(a,b)f_{yy}(a,b)-\bigl(f_{xy}(a,b)\bigr)^2.
\]

\medskip

\textbf{Second Derivative Test.}
\begin{itemize}
  \item If $D>0$ and $f_{xx}>0$, $f$ has a \textbf{local minimum}.
  \item If $D>0$ and $f_{xx}<0$, $f$ has a \textbf{local maximum}.
  \item If $D<0$, $f$ has a \textbf{saddle point}.
  \item If $D=0$, the test is \textbf{inconclusive}.
\end{itemize}

\subsection*{Solved Examples}

\textbf{Example 1: Classifying a Critical Point}

\textbf{Problem.}  
Find and classify the critical point of
\[
f(x,y)=x^2+y^2-4x-6y.
\]

\textbf{Solution.}

First partial derivatives:
\[
f_x=2x-4,\qquad f_y=2y-6.
\]
Setting both equal to zero gives $(x,y)=(2,3)$.

Second partial derivatives:
\[
f_{xx}=2,\qquad f_{yy}=2,\qquad f_{xy}=0.
\]
Thus
\[
D=2\cdot 2-0=4>0 \quad \text{and} \quad f_{xx}>0.
\]

\textbf{Conclusion.}  
The function has a local minimum at $(2,3)$.

\medskip

\textbf{Example 2: Saddle Point}

\textbf{Problem.}  
Classify the critical point of
\[
f(x,y)=x^2-y^2.
\]

\textbf{Solution.}

\[
f_x=2x,\qquad f_y=-2y.
\]
The only critical point is $(0,0)$.

Second partial derivatives:
\[
f_{xx}=2,\qquad f_{yy}=-2,\qquad f_{xy}=0.
\]
Thus
\[
D=2(-2)-0=-4<0.
\]

\textbf{Conclusion.}  
The point $(0,0)$ is a saddle point.

\subsection*{Applied Optimization}

\textbf{Example 3: Maximizing Revenue}

\textbf{Problem.}  
A company sells two products. Revenue (in dollars) is modeled by
\[
R(p_1,p_2)=180p_1-4p_1^2-2p_1p_2+120p_2-3p_2^2.
\]
Find the prices $p_1$ and $p_2$ that maximize revenue.

\textbf{Solution.}

First partial derivatives:
\[
R_{p_1}=180-8p_1-2p_2,\qquad R_{p_2}=120-2p_1-6p_2.
\]

Solve the system:
\[
180-8p_1-2p_2=0,\qquad 120-2p_1-6p_2=0.
\]

Solving yields
\[
(p_1,p_2)=\left(\frac{210}{11},\frac{150}{11}\right).
\]

Second derivatives:
\[
R_{p_1p_1}=-8,\quad R_{p_2p_2}=-6,\quad R_{p_1p_2}=-2.
\]
\[
D=(-8)(-6)-(-2)^2=44>0 \quad \text{and} \quad R_{p_1p_1}<0.
\]

\textbf{Conclusion.}  
Revenue is maximized at
\[
\left(\frac{210}{11},\frac{150}{11}\right).
\]

\subsection*{Practice Problems}

\begin{enumerate}
  \item Find and classify all critical points of
  \[
  f(x,y)=x^2+y^2-2x+4y.
  \]
  \workbox[3.0in]{}

  \item Find and classify all critical points of
  \[
  f(x,y)=x^2-y^2+4xy.
  \]
  \workbox[3.0in]{}

  \item A firm’s profit function is
  \[
  P(x,y)=50x+40y-x^2-2y^2-xy.
  \]
  Find the production levels $x$ and $y$ that maximize profit.
  \workbox[3.4in]{}

  \begin{answer}
  \textbf{Optimization, Practice Problems.}
  \begin{enumerate}
    \item Critical point $(1,-2)$; $D=4>0$ and $f_{xx}=2>0$, so local minimum.
    \item Critical point $(0,0)$; $D=-16<0$, so saddle point.
    \item $(x,y)=(12,7)$ gives a local maximum.
  \end{enumerate}
  \end{answer}
\end{enumerate}

\subsection*{Section Summary}

\begin{itemize}
  \item Optimization problems in two variables rely on partial derivatives.
  \item Critical points occur where both first partial derivatives are zero.
  \item The second derivative test classifies critical points.
  \item Saddle points behave like maxima in one direction and minima in another.
  \item These techniques are essential for business decision-making.
\end{itemize}
