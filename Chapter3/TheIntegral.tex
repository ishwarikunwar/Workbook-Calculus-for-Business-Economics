\chapter{The Integral}
\section{The Definite Integral}

\subsection*{Learning Objectives}
After completing this section, you should be able to:
\begin{itemize}
  \item Explain the meaning of the definite integral as an accumulation process.
  \item Interpret definite integrals in business and economic contexts.
  \item Understand the role of partitions and sums in defining an integral.
  \item Use definite-integral notation correctly.
  \item Distinguish between signed area and total accumulation.
\end{itemize}

\subsection*{Key Definitions and Concepts}

\textbf{Definite Integral.}  
The definite integral of a function $f(x)$ from $a$ to $b$ is a number that represents the total accumulation of the values of $f(x)$ over the interval $[a,b]$. It is denoted by
\[
\int_a^b f(x)\,dx.
\]

In business and economics, definite integrals often represent total cost, total revenue, total profit, or total change over a given interval.

\medskip

\textbf{Partition of an Interval.}  
A partition of $[a,b]$ divides the interval into $n$ subintervals:
\[
a=x_0<x_1<x_2<\cdots<x_n=b.
\]
Each subinterval has width $\Delta x$.

\medskip

\textbf{Riemann Sum.}  
A Riemann sum approximates the definite integral by summing function values multiplied by subinterval widths:
\[
\sum_{i=1}^n f(x_i^*)\,\Delta x,
\]
where $x_i^*$ is a sample point in the $i$th subinterval.

\medskip

\textbf{Limit of Riemann Sums.}  
The definite integral is defined as the limit of Riemann sums as the number of subintervals increases:
\[
\int_a^b f(x)\,dx
=
\lim_{n\to\infty}
\sum_{i=1}^n f(x_i^*)\,\Delta x.
\]

\medskip

\textbf{Signed Area.}  
If $f(x)$ is positive on an interval, the definite integral is positive.  
If $f(x)$ is negative, the definite integral is negative.  
Thus, the definite integral represents \emph{net accumulation}, not just geometric area.

\subsection*{Solved Examples}

\textbf{Example 1: Interpreting a Definite Integral}

\textbf{Problem.}  
Suppose $C'(x)$ represents the marginal cost (in dollars per unit) of producing $x$ units of a product. Explain the meaning of
\[
\int_{20}^{50} C'(x)\,dx.
\]

\textbf{Solution.}

The definite integral represents the total change in cost as production increases from 20 units to 50 units.

\textbf{Interpretation.}  
$\displaystyle \int_{20}^{50} C'(x)\,dx$ is the total additional cost incurred when production increases from 20 to 50 units.

\medskip

\textbf{Example 2: Accumulation of Revenue}

\textbf{Problem.}  
Let $R'(x)$ denote marginal revenue (in dollars per unit). What does
\[
\int_{0}^{100} R'(x)\,dx
\]
represent?

\textbf{Solution.}

This definite integral measures the accumulated revenue generated from selling the first 100 units.

\textbf{Interpretation.}  
The integral gives the total revenue earned from unit 0 through unit 100.

\medskip

\textbf{Example 3: Signed Area and Net Change}

\textbf{Problem.}  
Suppose $f(x)$ represents net cash flow (in dollars per day), where positive values indicate profit and negative values indicate loss. Interpret
\[
\int_{0}^{30} f(x)\,dx.
\]

\textbf{Solution.}

The integral represents the net cash flow over 30 days, accounting for both profits and losses.

\textbf{Interpretation.}  
Positive portions add to the total, while negative portions subtract, giving the overall net result.

\subsection*{Practice Problems}

\begin{enumerate}
  \item Let $P'(x)$ represent marginal profit (in dollars per unit).
  \begin{enumerate}
    \item Explain the meaning of $\displaystyle \int_{10}^{40} P'(x)\,dx$.
    \workbox[1.6in]{}
    \item What does a negative value of this integral indicate?
    \workbox[1.6in]{}
  \end{enumerate}

  \item Suppose $C'(x)$ is marginal cost.
  \begin{enumerate}
    \item Interpret $\displaystyle \int_{a}^{b} C'(x)\,dx$ in words.
    \workbox[1.6in]{}
    \item How does this relate to total cost?
    \workbox[1.6in]{}
  \end{enumerate}

  \item A function $f(x)$ represents the rate of change of inventory (units per week).
  \begin{enumerate}
    \item Explain the meaning of $\displaystyle \int_{0}^{8} f(x)\,dx$.
    \workbox[1.8in]{}
  \end{enumerate}

  \begin{answer}
  \textbf{The Definite Integral, Practice Problems.}
  \begin{enumerate}
    \item
    \begin{enumerate}
      \item The total change in profit as production increases from 10 to 40 units.
      \item A negative value indicates an overall loss in profit over that interval.
    \end{enumerate}

    \item
    \begin{enumerate}
      \item The total change in cost from $x=a$ to $x=b$.
      \item It represents the accumulated cost over that interval.
    \end{enumerate}

    \item The net change in inventory over 8 weeks.
  \end{enumerate}
  \end{answer}
\end{enumerate}

\subsection*{Section Summary}

\begin{itemize}
  \item The definite integral measures total accumulation over an interval.
  \item It is defined as the limit of Riemann sums.
  \item Definite integrals represent net change, not just geometric area.
  \item In business, integrals model total cost, revenue, profit, and inventory change.
  \item Understanding accumulation prepares for the Fundamental Theorem of Calculus.
\end{itemize}


\newpage

\section{The Fundamental Theorem of Calculus and Antidifferentiation}

\subsection*{Learning Objectives}
After completing this section, you should be able to:
\begin{itemize}
  \item State and explain the Fundamental Theorem of Calculus.
  \item Evaluate definite integrals using antiderivatives.
  \item Understand the relationship between differentiation and integration.
  \item Apply the Fundamental Theorem of Calculus in business contexts.
  \item Interpret definite integrals as net change.
\end{itemize}

\subsection*{Key Definitions and Concepts}

\textbf{Antiderivative.}  
A function $F(x)$ is an antiderivative of $f(x)$ if
\[
F'(x)=f(x).
\]
The collection of all antiderivatives of $f(x)$ is denoted by
\[
\int f(x)\,dx = F(x) + C,
\]
where $C$ is a constant.

\medskip

\textbf{Fundamental Theorem of Calculus (FTC).}  
If $f(x)$ is continuous on $[a,b]$ and $F(x)$ is any antiderivative of $f(x)$, then
\[
\int_a^b f(x)\,dx = F(b) - F(a).
\]

\medskip

\textbf{Interpretation.}  
The Fundamental Theorem of Calculus shows that differentiation and integration are inverse processes. It allows us to compute accumulated change using antiderivatives.

\medskip

\textbf{Business Interpretation.}  
If $f(x)$ represents a marginal quantity (such as marginal cost or marginal revenue), then the definite integral gives the total change over an interval.

\subsection*{Solved Examples}

\textbf{Example 1: Evaluating a Definite Integral Using FTC}

\textbf{Problem.}  
Evaluate
\[
\int_{2}^{5} (3x^2 - 4x + 1)\,dx.
\]

\textbf{Solution.}

\textbf{Step 1.} Find an antiderivative:
\[
F(x)=x^3 - 2x^2 + x.
\]

\textbf{Step 2.} Apply the Fundamental Theorem of Calculus:
\[
\int_{2}^{5} (3x^2 - 4x + 1)\,dx
= F(5) - F(2).
\]

\textbf{Step 3.} Evaluate:
\[
F(5)=125-50+5=80,
\quad
F(2)=8-8+2=2.
\]

\textbf{Conclusion.}
\[
\int_{2}^{5} (3x^2 - 4x + 1)\,dx = 78.
\]

\medskip

\textbf{Example 2: Total Cost from Marginal Cost}

\textbf{Problem.}  
Suppose marginal cost (in dollars per unit) is given by
\[
C'(x)=4x+20.
\]
Find the increase in total cost when production increases from 10 units to 25 units.

\textbf{Solution.}

\textbf{Step 1.} Integrate marginal cost:
\[
\int_{10}^{25} (4x+20)\,dx.
\]

\textbf{Step 2.} Find an antiderivative:
\[
F(x)=2x^2+20x.
\]

\textbf{Step 3.} Apply FTC:
\[
F(25)-F(10)
= \bigl(2(25)^2+20(25)\bigr)
- \bigl(2(10)^2+20(10)\bigr).
\]

\textbf{Step 4.} Evaluate:
\[
(1250+500)-(200+200)=1750-400=1350.
\]

\textbf{Interpretation.}  
The total cost increases by \$1,350 when production increases from 10 to 25 units.

\medskip

\textbf{Example 3: Net Change in Revenue}

\textbf{Problem.}  
Marginal revenue is given by
\[
R'(x)=60-3x.
\]
Find the change in revenue from selling 5 units to selling 15 units.

\textbf{Solution.}

\textbf{Step 1.} Set up the integral:
\[
\int_{5}^{15} (60-3x)\,dx.
\]

\textbf{Step 2.} Find an antiderivative:
\[
F(x)=60x-\frac{3}{2}x^2.
\]

\textbf{Step 3.} Apply FTC:
\[
F(15)-F(5).
\]

\textbf{Step 4.} Evaluate:
\[
\bigl(900-337.5\bigr)-\bigl(300-37.5\bigr)
=562.5-262.5=300.
\]

\textbf{Interpretation.}  
Revenue increases by \$300 when sales increase from 5 to 15 units.

\subsection*{Practice Problems}

\begin{enumerate}
  \item Evaluate the definite integral:
  \[
  \int_{1}^{4} (2x^2 - x)\,dx.
  \]
  \workbox[2.0in]{}

  \item Marginal cost is given by
  \[
  C'(x)=6x+10.
  \]
  Find the increase in total cost when production increases from 5 to 20 units.
  \workbox[2.2in]{}

  \item Marginal profit is given by
  \[
  P'(x)=50-2x.
  \]
  Find the change in profit when sales increase from 10 units to 30 units.
  \workbox[2.2in]{}

  \begin{answer}
  \textbf{FTC and Antidifferentiation, Practice Problems.}
  \begin{enumerate}
    \item $\displaystyle \int_{1}^{4} (2x^2 - x)\,dx = 30$.
    \item $\displaystyle \int_{5}^{20} (6x+10)\,dx = 1725$.
    \item $\displaystyle \int_{10}^{30} (50-2x)\,dx = 600$.
  \end{enumerate}
  \end{answer}
\end{enumerate}

\subsection*{Section Summary}

\begin{itemize}
  \item Antiderivatives reverse differentiation.
  \item The Fundamental Theorem of Calculus links derivatives and integrals.
  \item FTC allows efficient computation of definite integrals.
  \item In business, marginal functions integrate to total change.
  \item FTC is foundational for all applied integration techniques.
\end{itemize}


\newpage

\section{Antiderivatives of Formulas}

\subsection*{Learning Objectives}
After completing this section, you should be able to:
\begin{itemize}
  \item Use basic antiderivative rules to evaluate indefinite integrals.
  \item Apply the constant multiple and sum rules correctly.
  \item Use the power rule for antiderivatives.
  \item Integrate exponential and logarithmic functions.
  \item Include the constant of integration in all indefinite integrals.
\end{itemize}

\subsection*{Key Definitions and Concepts}

\textbf{Indefinite Integral.}  
An indefinite integral represents the family of all antiderivatives of a function and is written as
\[
\int f(x)\,dx = F(x) + C,
\]
where $C$ is an arbitrary constant.

\medskip

\textbf{Antiderivative Rules (Building Blocks).}

Let $f$ and $g$ be functions of $x$, and let $k$ and $n$ be constants.

\medskip

\textbf{Constant Multiple Rule.}
\[
\int k f(x)\,dx = k \int f(x)\,dx.
\]

\medskip

\textbf{Sum and Difference Rule.}
\[
\int \bigl(f(x)\pm g(x)\bigr)\,dx
=
\int f(x)\,dx \pm \int g(x)\,dx.
\]

\medskip

\textbf{Power Rule.}
For $n\neq -1$,
\[
\int x^n\,dx=\frac{x^{n+1}}{n+1}+C.
\]

\medskip

\textbf{Special Case: Constant Function.}
\[
\int k\,dx=kx+C.
\]

\medskip

\textbf{Exponential Functions.}
\[
\int e^x\,dx=e^x+C,
\qquad
\int a^x\,dx=\frac{a^x}{\ln a}+C \quad (a>0,\ a\neq1).
\]

\medskip

\textbf{Natural Logarithm.}
\[
\int \frac{1}{x}\,dx=\ln|x|+C.
\]

\subsection*{Solved Examples}

\textbf{Example 1: Using the Power Rule}

\textbf{Problem.}  
Evaluate
\[
\int (4x^3-6x+5)\,dx.
\]

\textbf{Solution.}

Apply the sum rule and power rule:
\[
\int 4x^3\,dx= x^4,
\quad
\int -6x\,dx=-3x^2,
\quad
\int 5\,dx=5x.
\]

\textbf{Final Answer.}
\[
x^4-3x^2+5x+C.
\]

\medskip

\textbf{Example 2: Exponential Functions}

\textbf{Problem.}  
Evaluate
\[
\int \bigl(3e^x+2\cdot5^x\bigr)\,dx.
\]

\textbf{Solution.}

Apply the constant multiple and exponential rules:
\[
\int 3e^x\,dx=3e^x,
\qquad
\int 2\cdot5^x\,dx=\frac{2\cdot5^x}{\ln 5}.
\]

\textbf{Final Answer.}
\[
3e^x+\frac{2\cdot5^x}{\ln 5}+C.
\]

\medskip

\textbf{Example 3: Logarithmic Antiderivative}

\textbf{Problem.}  
Evaluate
\[
\int \frac{7}{x}\,dx.
\]

\textbf{Solution.}

Use the constant multiple rule:
\[
\int \frac{7}{x}\,dx=7\ln|x|+C.
\]

\subsection*{Practice Problems}

\begin{enumerate}
  \item Evaluate:
  \[
  \int (6x^2-4x+9)\,dx.
  \]
  \workbox[2.2in]{}

  \item Evaluate:
  \[
  \int (2e^x-3\cdot4^x)\,dx.
  \]
  \workbox[2.2in]{}

  \item Evaluate:
  \[
  \int \frac{5}{x}\,dx.
  \]
  \workbox[1.8in]{}

  \item Evaluate:
  \[
  \int (x^{-3}+2x^{1/2})\,dx.
  \]
  \workbox[2.4in]{}

  \begin{answer}
  \textbf{Antiderivatives of Formulas, Practice Problems.}
  \begin{enumerate}
    \item $2x^3-2x^2+9x+C$.
    \item $2e^x-\dfrac{3\cdot4^x}{\ln4}+C$.
    \item $5\ln|x|+C$.
    \item $-\dfrac{1}{2}x^{-2}+\dfrac{4}{3}x^{3/2}+C$.
  \end{enumerate}
  \end{answer}
\end{enumerate}

\subsection*{Section Summary}

\begin{itemize}
  \item Antiderivative rules allow efficient computation of indefinite integrals.
  \item The power rule applies to most algebraic functions.
  \item Exponential and logarithmic functions have special antiderivative formulas.
  \item The constant of integration represents a family of functions.
  \item These rules form the foundation for all integration techniques.
\end{itemize}


\newpage

\section{Substitution}

\subsection*{Learning Objectives}
After completing this section, you should be able to:
\begin{itemize}
  \item Recognize integrals that require substitution.
  \item Apply substitution to evaluate indefinite integrals.
  \item Correctly change variables and differentials.
  \item Reverse the chain rule through integration.
  \item Use substitution in simple business and economic models.
\end{itemize}

\subsection*{Key Definitions and Concepts}

\textbf{Substitution Method.}  
Substitution is a technique for evaluating integrals by changing variables to simplify the integrand. It is based on reversing the chain rule for differentiation.

\medskip

\textbf{Basic Idea.}  
If an integrand contains a function and its derivative (or something close to it), substitution may simplify the integral.

\medskip

\textbf{Procedure for Substitution.}
\begin{enumerate}
  \item Choose a substitution $u=g(x)$.
  \item Compute $du=g'(x)\,dx$.
  \item Rewrite the integral entirely in terms of $u$.
  \item Integrate with respect to $u$.
  \item Substitute back in terms of $x$.
\end{enumerate}

\medskip

\textbf{Why Substitution Works.}  
Substitution reverses the chain rule:
\[
\frac{d}{dx}F(g(x)) = F'(g(x))g'(x).
\]
Integration undoes this process.

\subsection*{Solved Examples}

\textbf{Example 1: Basic Substitution}

\textbf{Problem.}  
Evaluate
\[
\int 2x(3x^2+1)^4\,dx.
\]

\textbf{Solution.}

\textbf{Step 1.} Let $u=3x^2+1$.

\textbf{Step 2.} Compute $du=6x\,dx$, so $\frac{1}{3}du=2x\,dx$.

\textbf{Step 3.} Rewrite the integral:
\[
\int (3x^2+1)^4(2x\,dx)
=
\frac{1}{3}\int u^4\,du.
\]

\textbf{Step 4.} Integrate:
\[
\frac{1}{3}\cdot\frac{u^5}{5}+C=\frac{u^5}{15}+C.
\]

\textbf{Step 5.} Substitute back:
\[
\frac{(3x^2+1)^5}{15}+C.
\]

\medskip

\textbf{Example 2: Exponential Function}

\textbf{Problem.}  
Evaluate
\[
\int e^{5x-2}\,dx.
\]

\textbf{Solution.}

\textbf{Step 1.} Let $u=5x-2$.

\textbf{Step 2.} Then $du=5\,dx$, so $\frac{1}{5}du=dx$.

\textbf{Step 3.} Rewrite and integrate:
\[
\frac{1}{5}\int e^u\,du=\frac{1}{5}e^u+C.
\]

\textbf{Final Answer.}
\[
\frac{1}{5}e^{5x-2}+C.
\]

\medskip

\textbf{Example 3: Logarithmic Form}

\textbf{Problem.}  
Evaluate
\[
\int \frac{4x}{x^2+9}\,dx.
\]

\textbf{Solution.}

\textbf{Step 1.} Let $u=x^2+9$.

\textbf{Step 2.} Then $du=2x\,dx$, so $2\,du=4x\,dx$.

\textbf{Step 3.} Rewrite and integrate:
\[
\int \frac{4x}{x^2+9}\,dx
=
2\int \frac{1}{u}\,du
=
2\ln|u|+C.
\]

\textbf{Final Answer.}
\[
2\ln(x^2+9)+C.
\]

\subsection*{Practice Problems}

\begin{enumerate}
  \item Evaluate:
  \[
  \int x(2x^2+5)^3\,dx.
  \]
  \workbox[2.6in]{}

  \item Evaluate:
  \[
  \int e^{4x+1}\,dx.
  \]
  \workbox[2.0in]{}

  \item Evaluate:
  \[
  \int \frac{6x}{x^2+4}\,dx.
  \]
  \workbox[2.2in]{}

  \begin{answer}
  \textbf{Substitution, Practice Problems.}
  \begin{enumerate}
    \item $\dfrac{(2x^2+5)^4}{8}+C.$
    \item $\dfrac{1}{4}e^{4x+1}+C.$
    \item $3\ln(x^2+4)+C.$
  \end{enumerate}
  \end{answer}
\end{enumerate}

\subsection*{Section Summary}

\begin{itemize}
  \item Substitution simplifies integrals by changing variables.
  \item It reverses the chain rule for differentiation.
  \item Successful substitution identifies a function and its derivative.
  \item Algebraic substitution prepares for applied integration problems.
  \item Substitution is a core technique for evaluating complex integrals.
\end{itemize}


\newpage

\section{Additional Integration Techniques}

\subsection*{Learning Objectives}
After completing this section, you should be able to:
\begin{itemize}
  \item Apply integration by parts to evaluate integrals of products.
  \item Choose appropriate functions for $u$ and $dv$.
  \item Use the integration by parts formula for indefinite integrals.
  \item Recognize integrals that match standard integral formulas.
  \item Evaluate selected algebraic integrals using known results.
\end{itemize}

\subsection*{Key Definitions and Concepts}

\textbf{Integration by Parts.}  
Integration by parts is a technique used to evaluate integrals of products of functions. It is based on the product rule for differentiation.

\medskip

\textbf{Formula for Integration by Parts.}
If $u=u(x)$ and $v=v(x)$ are differentiable functions, then
\[
\int u\,dv = uv - \int v\,du.
\]

\medskip

\textbf{Choosing $u$ and $dv$.}  
The goal is to choose $u$ so that its derivative $du$ is simpler than $u$, while $dv$ is easy to integrate.

\medskip

\textbf{Definite Integrals.}  
For definite integrals, integration by parts becomes
\[
\int_a^b u\,dv = \left.uv\right|_a^b - \int_a^b v\,du.
\]

\medskip

\textbf{Standard Integral Forms.}  
Some integrals occur frequently and have known formulas, such as:
\[
\int \frac{1}{x^2-a^2}\,dx=\frac{1}{2a}\ln\left|\frac{x-a}{x+a}\right|+C,
\]
\[
\int \frac{1}{\sqrt{x^2+a^2}}\,dx=\ln\left|x+\sqrt{x^2+a^2}\right|+C.
\]

\subsection*{Solved Examples}

\textbf{Example 1: Integration by Parts}

\textbf{Problem.}  
Evaluate
\[
\int x\ln x\,dx.
\]

\textbf{Solution.}

\textbf{Step 1.} Choose
\[
u=\ln x \quad\Rightarrow\quad du=\frac{1}{x}\,dx,
\]
\[
dv=x\,dx \quad\Rightarrow\quad v=\frac{x^2}{2}.
\]

\textbf{Step 2.} Apply the formula:
\[
\int x\ln x\,dx
=
\frac{x^2}{2}\ln x-\int \frac{x^2}{2}\cdot\frac{1}{x}\,dx.
\]

\textbf{Step 3.} Simplify and integrate:
\[
\frac{x^2}{2}\ln x-\frac{1}{2}\int x\,dx
=
\frac{x^2}{2}\ln x-\frac{x^2}{4}+C.
\]

\medskip

\textbf{Example 2: Definite Integral Using Integration by Parts}

\textbf{Problem.}  
Evaluate
\[
\int_1^e \ln x\,dx.
\]

\textbf{Solution.}

Let $u=\ln x$ and $dv=dx$. Then $du=\frac{1}{x}dx$ and $v=x$.

\[
\int_1^e \ln x\,dx
=
\left.x\ln x\right|_1^e-\int_1^e 1\,dx.
\]

\[
= (e\cdot1-0)-(e-1)=1.
\]

\textbf{Conclusion.}
\[
\int_1^e \ln x\,dx=1.
\]

\medskip

\textbf{Example 3: Using a Standard Integral Formula}

\textbf{Problem.}  
Evaluate
\[
\int \frac{1}{x^2-9}\,dx.
\]

\textbf{Solution.}

This matches the standard form with $a=3$:
\[
\int \frac{1}{x^2-9}\,dx
=
\frac{1}{6}\ln\left|\frac{x-3}{x+3}\right|+C.
\]

\subsection*{Practice Problems}

\begin{enumerate}
  \item Evaluate:
  \[
  \int x\ln x\,dx.
  \]
  \workbox[2.8in]{}

  \item Evaluate:
  \[
  \int_1^e \ln x\,dx.
  \]
  \workbox[2.0in]{}

  \item Evaluate:
  \[
  \int \frac{1}{x^2-16}\,dx.
  \]
  \workbox[2.2in]{}

  \begin{answer}
  \textbf{Additional Integration Techniques, Practice Problems.}
  \begin{enumerate}
    \item $\dfrac{x^2}{2}\ln x-\dfrac{x^2}{4}+C.$
    \item $1.$
    \item $\dfrac{1}{8}\ln\left|\dfrac{x-4}{x+4}\right|+C.$
  \end{enumerate}
  \end{answer}
\end{enumerate}

\subsection*{Section Summary}

\begin{itemize}
  \item Integration by parts evaluates integrals of products.
  \item It is based on reversing the product rule.
  \item Choosing $u$ appropriately simplifies the problem.
  \item Some integrals follow known algebraic formulas.
  \item These techniques extend the range of integrals we can evaluate.
\end{itemize}


\newpage

\section{Area, Volume, and Average Value}

\subsection*{Learning Objectives}
After completing this section, you should be able to:
\begin{itemize}
  \item Compute the area under a curve using definite integrals.
  \item Find the area between two curves.
  \item Compute volumes of solids formed by rotating a region about the $x$-axis.
  \item Find the average value of a function on an interval.
  \item Interpret area, volume, and average value in applied contexts.
\end{itemize}

\subsection*{Key Definitions and Concepts}

\textbf{Area Under a Curve.}  
If $f(x)\ge 0$ on $[a,b]$, the area between the graph of $f(x)$ and the $x$-axis is
\[
\text{Area}=\int_a^b f(x)\,dx.
\]

\medskip

\textbf{Area Between Two Curves.}  
If $f(x)\ge g(x)$ on $[a,b]$, then the area between the curves is
\[
\text{Area}=\int_a^b \bigl(f(x)-g(x)\bigr)\,dx.
\]

\medskip

\textbf{Volume of a Solid of Revolution.}  
Consider the region bounded by the graph of $f(x)$, the $x$-axis, and the vertical lines $x=a$ and $x=b$, where $f(x)\ge0$ on $[a,b]$.

If this region is rotated about the $x$-axis, it forms a three-dimensional solid.

\medskip

Partition the interval $[a,b]$ into subintervals of width $\Delta x$. Each subinterval generates a thin cylindrical disk with:
\begin{itemize}
  \item radius $f(x)$,
  \item height $\Delta x$.
\end{itemize}

The volume of one disk is approximately
\[
\pi\bigl(f(x)\bigr)^2\Delta x.
\]

Adding the volumes of all disks gives a Riemann sum:
\[
\sum \pi\bigl(f(x)\bigr)^2\Delta x.
\]

Taking the limit as $\Delta x\to0$ yields the exact volume.

\medskip

\textbf{Disk Method Formula.}
\[
V=\int_a^b \pi\bigl(f(x)\bigr)^2\,dx.
\]

\medskip

\textbf{Average Value of a Function.}  
The average value of $f(x)$ on $[a,b]$ is
\[
f_{\text{avg}}=\frac{1}{b-a}\int_a^b f(x)\,dx.
\]

\subsection*{Solved Examples}

\textbf{Example 1: Area Under a Curve}

\textbf{Problem.}  
Find the area under the curve $f(x)=2x$ from $x=0$ to $x=5$.

\textbf{Solution.}

\[
\text{Area}=\int_0^5 2x\,dx
=\left[x^2\right]_0^5
=25.
\]

\medskip

\textbf{Example 2: Area Between Two Curves}

\textbf{Problem.}  
Find the area between $f(x)=x$ and $g(x)=x^2$ on $[0,1]$.

\textbf{Solution.}

Since $x\ge x^2$ on $[0,1]$,
\[
\text{Area}=\int_0^1 (x-x^2)\,dx
=\left[\frac{x^2}{2}-\frac{x^3}{3}\right]_0^1
=\frac{1}{6}.
\]

\medskip

\textbf{Example 3: Volume Using the Disk Method}

\textbf{Problem.}  
Find the volume of the solid obtained by rotating $f(x)=x$ about the $x$-axis on $[0,2]$.

\textbf{Solution.}

Using the disk method,
\[
V=\int_0^2 \pi x^2\,dx
=\pi\left[\frac{x^3}{3}\right]_0^2
=\frac{8\pi}{3}.
\]

\medskip

\textbf{Example 4: Average Value}

\textbf{Problem.}  
Find the average value of $f(x)=3x^2$ on $[0,2]$.

\textbf{Solution.}

\[
f_{\text{avg}}
=\frac{1}{2-0}\int_0^2 3x^2\,dx
=\frac{1}{2}\left[x^3\right]_0^2
=4.
\]

\subsection*{Practice Problems}

\begin{enumerate}
  \item Find the area under $f(x)=4x$ from $x=0$ to $x=3$.
  \workbox[2.2in]{}

  \item Find the area between $f(x)=2x$ and $g(x)=x^2$ on $[0,2]$.
  \workbox[2.6in]{}

  \item Find the volume of the solid obtained by rotating $f(x)=x^2$ about the $x$-axis on $[0,1]$.
  \workbox[2.8in]{}

  \item Find the average value of $f(x)=x^2+2$ on $[1,3]$.
  \workbox[2.4in]{}

  \begin{answer}
  \textbf{Area, Volume, and Average Value, Practice Problems.}
  \begin{enumerate}
    \item $18.$
    \item $\dfrac{4}{3}.$
    \item $\dfrac{\pi}{5}.$
    \item $\dfrac{13}{3}.$
  \end{enumerate}
  \end{answer}
\end{enumerate}

\subsection*{Section Summary}

\begin{itemize}
  \item Definite integrals compute areas and accumulated quantities.
  \item Area between curves is found by integrating the difference.
  \item Volumes of solids of revolution are computed using the disk method.
  \item Average value represents the mean value of a function on an interval.
  \item These concepts have broad applications in business and economics.
\end{itemize}


\newpage

\section{Applications to Business}

\subsection*{Learning Objectives}
After completing this section, you should be able to:
\begin{itemize}
  \item Identify demand and supply functions.
  \item Determine equilibrium price and quantity.
  \item Compute consumer and producer surplus using definite integrals.
  \item Interpret surplus values as gains from trade.
  \item Compute the present value of a continuous income stream.
  \item Compare investments using present value.
\end{itemize}

\subsection*{Key Definitions and Concepts}

\textbf{Demand and Supply.}  
A demand function $p=d(q)$ gives the price consumers are willing to pay for a given quantity $q$ and is typically decreasing.

A supply function $p=s(q)$ gives the price producers are willing to accept for a given quantity $q$ and is typically increasing.

\medskip

\textbf{Equilibrium.}  
The equilibrium point $(q^*,p^*)$ occurs where demand equals supply:
\[
d(q^*)=s(q^*)=p^*.
\]

\medskip

\textbf{Consumer Surplus.}  
Consumer surplus is the total benefit consumers receive by paying less than the maximum price they are willing to pay.

\medskip

\textbf{Producer Surplus.}  
Producer surplus is the total benefit producers receive by selling at a price higher than the minimum price they are willing to accept.

\medskip

\textbf{Surplus Formulas.}  
Given demand $p=d(q)$, supply $p=s(q)$, and equilibrium $(q^*,p^*)$:
\[
\text{Consumer Surplus}=\int_0^{q^*} d(q)\,dq - p^*q^*,
\]
\[
\text{Producer Surplus}=p^*q^*-\int_0^{q^*} s(q)\,dq.
\]

The sum of consumer and producer surplus represents the \textbf{total gains from trade}.

\subsection*{Solved Examples}

\textbf{Example 1: Finding Equilibrium}

\textbf{Problem.}  
Suppose demand and supply are given by
\[
d(q)=100-2q,
\qquad
s(q)=20+q.
\]
Find the equilibrium price and quantity.

\textbf{Solution.}

Set demand equal to supply:
\[
100-2q=20+q.
\]

Solving,
\[
3q=80 \quad \Rightarrow \quad q^*=\frac{80}{3}.
\]

Substitute into either function:
\[
p^*=100-2\left(\frac{80}{3}\right)=\frac{140}{3}.
\]

\medskip

\textbf{Example 2: Consumer Surplus}

\textbf{Problem.}  
Using the functions from Example 1, find the consumer surplus.

\textbf{Solution.}

\[
\text{CS}
=
\int_0^{80/3} (100-2q)\,dq - \frac{140}{3}\cdot\frac{80}{3}.
\]

\[
=
\left[100q-q^2\right]_0^{80/3}-\frac{11200}{9}
=
\frac{17600}{9}-\frac{11200}{9}
=
\frac{6400}{9}.
\]

\medskip

\textbf{Example 3: Producer Surplus}

\textbf{Problem.}  
Using the same functions, find the producer surplus.

\textbf{Solution.}

\[
\text{PS}
=
\frac{140}{3}\cdot\frac{80}{3}-\int_0^{80/3}(20+q)\,dq.
\]

\[
=
\frac{11200}{9}-\left[20q+\frac{q^2}{2}\right]_0^{80/3}
=
\frac{11200}{9}-\frac{8000}{9}
=
\frac{3200}{9}.
\]

\subsection*{Continuous Income Stream}

\textbf{Compound Interest Review.}  
Let $P$ be the principal, $r$ the annual interest rate, and $t$ the time in years.

\begin{itemize}
  \item Compounded $n$ times per year:
  \[
  A(t)=P\left(1+\frac{r}{n}\right)^{nt}.
  \]
  \item Compounded continuously:
  \[
  A(t)=Pe^{rt}.
  \]
\end{itemize}

\medskip

\textbf{Continuous Income Stream.}  
Suppose money earns interest at an annual rate $r$, compounded continuously. Let $F(t)$ be a continuous income function (in dollars per year) received between time $0$ and time $T$.

Each small payment must be discounted back to the present using the exponential factor.

\medskip

\textbf{Present Value Formula.}
\[
\text{PV}=\int_0^T F(t)e^{-rt}\,dt.
\]

\medskip

\textbf{Future Value.}  
Once the present value is found, the future value at time $T$ is
\[
\text{FV}=\text{PV}\,e^{rT}.
\]

\medskip

\textbf{Interpretation.}  
Present value allows us to compare different investments by measuring their worth in today’s dollars.

\subsection*{Practice Problems}

\begin{enumerate}
  \item Demand and supply are given by
  \[
  d(q)=60-q,
  \qquad
  s(q)=q+10.
  \]
  \begin{enumerate}
    \item Find the equilibrium quantity and price.
    \workbox[2.2in]{}
    \item Compute the consumer surplus.
    \workbox[2.6in]{}
    \item Compute the producer surplus.
    \workbox[2.6in]{}
  \end{enumerate}

  \item A continuous income stream pays \$1000 per year for 5 years.  
  The interest rate is 6\% per year, compounded continuously.
  \begin{enumerate}
    \item Find the present value of the income stream.
    \workbox[2.8in]{}
    \item Find the future value at the end of 5 years.
    \workbox[2.6in]{}
  \end{enumerate}

  \begin{answer}
  \textbf{Applications to Business, Practice Problems.}
  \begin{enumerate}
    \item
    \begin{enumerate}
      \item $q^*=25,\; p^*=35.$
      \item $\text{CS}=\dfrac{625}{2}.$
      \item $\text{PS}=\dfrac{625}{2}.$
    \end{enumerate}
    \item
    \begin{enumerate}
      \item $\text{PV}=\dfrac{1000}{0.06}\left(1-e^{-0.3}\right).$
      \item $\text{FV}=\dfrac{1000}{0.06}\left(e^{0.3}-1\right).$
    \end{enumerate}
  \end{enumerate}
  \end{answer}
\end{enumerate}

\subsection*{Section Summary}

\begin{itemize}
  \item Equilibrium occurs where demand equals supply.
  \item Consumer and producer surplus measure gains from trade.
  \item Definite integrals compute surplus as areas.
  \item Continuous income streams are valued using discounted integrals.
  \item Present value compares investments in today’s dollars.
\end{itemize}


\newpage

\section{Differential Equations}

\subsection*{Learning Objectives}
After completing this section, you should be able to:
\begin{itemize}
  \item Understand what a differential equation represents.
  \item Model real-world situations using differential equations.
  \item Check whether a given function satisfies a differential equation.
  \item Solve separable differential equations.
  \item Apply differential equations to growth models.
\end{itemize}

\subsection*{Key Definitions and Concepts}

\textbf{Differential Equation.}  
A differential equation is an equation involving a function and one or more of its derivatives. Differential equations model relationships between quantities and their rates of change.

\medskip

\textbf{Solution of a Differential Equation.}  
A solution of a differential equation is a function that satisfies the equation when substituted into it.

\medskip

\textbf{Initial Condition.}  
An initial condition specifies the value of the function at a particular point and allows us to determine a unique solution.

\subsection*{Modeling with Differential Equations}

\textbf{Example 1: Bank Balance Model}

\textbf{Problem.}  
A bank pays 2\% interest on a certificate of deposit but charges a \$20 annual fee. Write a differential equation for the balance $B(t)$.

\textbf{Solution.}

The balance changes due to:
\begin{itemize}
  \item interest earned: $0.02B(t)$ dollars per year,
  \item fee charged: \$20 per year.
\end{itemize}

Thus,
\[
B'(t)=0.02B(t)-20.
\]

\textbf{Interpretation.}  
The rate of change of the balance depends on the current balance and a fixed annual fee.

\subsection*{Checking Solutions of Differential Equations}

\textbf{Example 2: Verifying a Solution}

\textbf{Problem.}  
Check whether $y=x^2+5$ is a solution of
\[
y'+y=x^2+7.
\]

\textbf{Solution.}

Compute the derivative:
\[
y'=2x.
\]

Substitute into the equation:
\[
y'+y=2x+(x^2+5)=x^2+2x+5.
\]

Since this expression is not equal to $x^2+7$ for all $x$, $y=x^2+5$ is \emph{not} a solution.

\medskip

\textbf{Example 3: Checking Another Solution}

\textbf{Problem.}  
Check whether $y=x+\dfrac{5}{x}$ satisfies
\[
y'+\frac{y}{x}=2.
\]

\textbf{Solution.}

Compute the derivative:
\[
y'=1-\frac{5}{x^2}.
\]

Substitute:
\[
y'+\frac{y}{x}
=
\left(1-\frac{5}{x^2}\right)
+\frac{1}{x}\left(x+\frac{5}{x}\right)
=
1-\frac{5}{x^2}+1+\frac{5}{x^2}
=2.
\]

Thus, $y=x+\dfrac{5}{x}$ is a solution.

\subsection*{Separable Differential Equations}

\textbf{Definition.}  
A differential equation is separable if it can be written in the form
\[
g(y)\,dy=f(x)\,dx.
\]

\medskip

\textbf{Example 4: Solving a Separable Equation}

\textbf{Problem.}  
Find the general solution of
\[
\frac{dy}{dx}=\frac{6x+1}{2y}.
\]

\textbf{Solution.}

Rewrite:
\[
2y\,dy=(6x+1)\,dx.
\]

Integrate both sides:
\[
\int 2y\,dy=\int(6x+1)\,dx.
\]

\[
y^2=3x^2+x+C.
\]

\subsection*{Models of Growth}

\textbf{Unlimited Growth.}  
If a quantity grows at a rate proportional to its size, it can be modeled by
\[
\frac{dy}{dt}=ry,
\]
where $r$ is a constant.

\medskip

\textbf{Example 5: Population Growth}

\textbf{Problem.}  
A population grows at 8\% per year. If the current population is 5,000, find a formula for the population after $t$ years.

\textbf{Solution.}

\[
\frac{dy}{dt}=0.08y.
\]

Separate variables:
\[
\frac{1}{y}dy=0.08\,dt.
\]

Integrate:
\[
\ln|y|=0.08t+C.
\]

Exponentiate:
\[
y=Ae^{0.08t}.
\]

Apply the initial condition $y(0)=5000$:
\[
5000=A.
\]

Thus,
\[
y=5000e^{0.08t}.
\]

\medskip

\textbf{Limited Growth.}  
If growth slows as a quantity approaches a maximum value $M$, it can be modeled by
\[
\frac{dy}{dt}=k(M-y),
\]
where $k$ is a constant.

\medskip

\textbf{Logistic Growth.}  
If growth depends on both the current size and the distance from a maximum value $M$, the model is
\[
\frac{dy}{dt}=ry\left(1-\frac{y}{M}\right).
\]

The solution has the form
\[
y=\frac{M}{1+Ae^{-rt}}.
\]

\subsection*{Practice Problems}

\begin{enumerate}
  \item Write a differential equation for a bank account that earns 3\% interest per year and charges a \$15 annual fee.
  \workbox[2.2in]{}

  \item Check whether $y=x^2-4$ is a solution of $y'+y=x^2-2$.
  \workbox[2.2in]{}

  \item Solve the separable differential equation:
  \[
  \frac{dy}{dx}=\frac{4x}{y}.
  \]
  \workbox[2.6in]{}

  \item A population grows at a rate proportional to its size with growth rate 5\%. Write the differential equation and general solution.
  \workbox[2.6in]{}

  \item A population has limited growth with maximum size 10,000 and growth constant $k=0.02$. Write the differential equation.
  \workbox[2.2in]{}

  \begin{answer}
  \textbf{Differential Equations, Practice Problems.}
  \begin{enumerate}
    \item $B'(t)=0.03B(t)-15.$
    \item Not a solution.
    \item $y^2=2x^2+C.$
    \item $\dfrac{dy}{dt}=0.05y,\quad y=Ae^{0.05t}.$
    \item $\dfrac{dy}{dt}=0.02(10000-y).$
  \end{enumerate}
  \end{answer}
\end{enumerate}

\subsection*{Section Summary}

\begin{itemize}
  \item Differential equations model relationships involving rates of change.
  \item Solutions are functions that satisfy the equation.
  \item Separable equations can be solved by integration.
  \item Growth models describe population and financial behavior.
  \item Differential equations connect calculus to real-world dynamics.
\end{itemize}




