\chapter{The Derivative}

\section{Limits and Continuity}

\subsection*{Learning Objectives}
After completing this section, you should be able to:
\begin{itemize}
  \item Evaluate limits of functions algebraically.
  \item Interpret limits in the context of business and economics.
  \item Determine whether a function is continuous at a given point.
  \item Identify and classify types of discontinuities.
  \item Explain the practical meaning of continuity in business models.
\end{itemize}

\subsection*{Key Definitions and Concepts}

\textbf{Limit.}  
The limit of a function $f(x)$ as $x$ approaches a value $a$ is the number that $f(x)$ approaches as $x$ gets close to $a$, provided this value exists.
\[
\lim_{x \to a} f(x) = L
\]
In business applications, limits are often used to analyze behavior \emph{near} a certain level of output, price, or time, rather than exactly at that value.

\medskip

\textbf{One-Sided Limits.}  
If a function approaches different values from the left and the right of $a$, we consider one-sided limits:
\[
\lim_{x \to a^-} f(x), \qquad \lim_{x \to a^+} f(x)
\]
A two-sided limit exists only if these one-sided limits are equal.

\medskip

\textbf{Continuity.}  
A function $f(x)$ is continuous at $x = a$ if all three of the following conditions are satisfied:
\begin{enumerate}
  \item $f(a)$ is defined.
  \item $\lim_{x \to a} f(x)$ exists.
  \item $\lim_{x \to a} f(x) = f(a)$.
\end{enumerate}
In business models, continuity means that small changes in input produce small changes in output, which is often a realistic assumption for cost, revenue, and profit functions.

\subsection*{Solved Examples}

\textbf{Example 1: Evaluating a Limit Algebraically}

\textbf{Problem.}  
Suppose the cost (in dollars) of producing $x$ units of a product is given by
\[
C(x) = 3x^2 + 5x + 200.
\]
Find
\[
\lim_{x \to 10} C(x)
\]
and interpret the result.

\textbf{Solution.}

Step 1: Recognize that $C(x)$ is a polynomial.  
Polynomials are continuous for all real values of $x$.

Step 2: Use direct substitution to evaluate the limit.
\[
\lim_{x \to 10} C(x) = C(10)
\]

Step 3: Compute the value.
\[
C(10) = 3(10)^2 + 5(10) + 200 = 550
\]

\textbf{Interpretation.}  
As production approaches 10 units, the cost approaches \$550. Producing close to 10 units will result in a cost close to \$550.

\medskip

\textbf{Example 2: A Limit That Does Not Exist}

\textbf{Problem.}  
A company's pricing model is defined by
\[
p(x) =
\begin{cases}
50, & x < 100, \\
65, & x \ge 100,
\end{cases}
\]
where $x$ is the number of units ordered. Determine whether
\[
\lim_{x \to 100} p(x)
\]
exists.

\textbf{Solution.}

Step 1: Evaluate the left-hand limit.
\[
\lim_{x \to 100^-} p(x) = 50
\]

Step 2: Evaluate the right-hand limit.
\[
\lim_{x \to 100^+} p(x) = 65
\]

Step 3: Compare the one-sided limits.  
Since $50 \neq 65$, the two-sided limit does not exist.

\textbf{Interpretation.}  
The price jumps at 100 units, indicating a quantity discount threshold. The pricing function is not continuous at $x = 100$.

\medskip

\textbf{Example 3: Continuity at a Point}

\textbf{Problem.}  
Let profit (in dollars) be given by
\[
P(x) =
\begin{cases}
4x - 100, & x \le 50, \\
3x + 50, & x > 50.
\end{cases}
\]
Determine whether $P(x)$ is continuous at $x = 50$.

\textbf{Solution.}

Step 1: Evaluate $P(50)$.
\[
P(50) = 4(50) - 100 = 100
\]

Step 2: Compute the left-hand limit.
\[
\lim_{x \to 50^-} P(x) = 100
\]

Step 3: Compute the right-hand limit.
\[
\lim_{x \to 50^+} P(x) = 200
\]

Step 4: Compare the results.  
Since the left-hand and right-hand limits are not equal, the limit does not exist.

\textbf{Conclusion.}  
The profit function is not continuous at $x = 50$, indicating an abrupt change in profit behavior at that production level.

\subsection*{Practice Problems}

\begin{enumerate}
  \item A revenue function is given by $R(x) = 12x + 80$.
  \begin{enumerate}
    \item Evaluate $\displaystyle \lim_{x \to 25} R(x)$.
    \workbox[1.0in]{}
    \item Explain the meaning of the result in context.
    \workbox[1.4in]{}
  \end{enumerate}
  \begin{answer}
  \textbf{Limits and Continuity, Problem 1.}
  \begin{enumerate}
      \item $\displaystyle \lim_{x \to 25} R(x) = 380$.
      \item As output approaches 25 units, revenue approaches \$380, so revenue near 25 units will be close to \$380.
   \end{enumerate}
  \end{answer}  
 \item Consider the function
  \[
  C(x) =
  \begin{cases}
  500, & x < 40, \\
  500 + 10(x-40), & x \ge 40.
  \end{cases}
  \]
  \begin{enumerate}
    \item Find $\displaystyle \lim_{x \to 40^-} C(x)$.
    \workbox[1.0in]{}
    \item Find $\displaystyle \lim_{x \to 40^+} C(x)$.
    \workbox[1.0in]{}
    \item Determine whether $C(x)$ is continuous at $x=40$. Justify using the definition of continuity.
    \workbox[2.0in]{}
  \end{enumerate}
  \begin{answer}
  \textbf{Limits and Continuity, Problem 2.}
  \begin{enumerate}
    \item $\displaystyle \lim_{x \to 40^-} C(x) = 500$.
    \item $\displaystyle \lim_{x \to 40^+} C(x) = 500$.
    \item $C(40)=500$ and $\lim_{x\to 40}C(x)=500$, so $C$ is continuous at $x=40$.
  \end{enumerate}
  \end{answer}
\item A demand function is defined for all $x \neq 60$.
  \begin{enumerate}
    \item Explain why $\displaystyle \lim_{x \to 60} f(x)$ may exist even if $f(60)$ is not defined.
    \workbox[1.6in]{}
\end{enumerate}
\begin{answer}
  \textbf{Limits and Continuity, Problem 3.}
  A limit depends on the values of the function near the point, not necessarily the value at the point. The limit can exist even if $f(60)$ is undefined.
\end{answer}
\end{enumerate}

\subsection*{Section Summary}

\begin{itemize}
  \item Limits describe how a function behaves as the input approaches a particular value.
  \item A limit may exist even if the function is not defined at that value.
  \item Continuity requires that the function value and the limit agree at a point.
  \item Discontinuities often represent abrupt changes in business policies, pricing, or production.
  \item Understanding limits and continuity is essential for modeling smooth changes in cost, revenue, and profit.
\end{itemize}

\newpage
\section{The Derivative}

\subsection*{Learning Objectives}
After completing this section, you should be able to:
\begin{itemize}
  \item Explain the derivative as an instantaneous rate of change and as the slope of a tangent line.
  \item Compute a derivative at a point using the limit definition.
  \item Interpret derivatives in business and economics contexts (e.g., marginal cost, marginal revenue, marginal profit).
  \item Approximate derivatives from tables or graphs using nearby average rates of change.
  \item Determine whether a function is differentiable at a point based on its behavior.
\end{itemize}

\subsection*{Key Definitions and Concepts}

\textbf{Derivative at a Point.}
The derivative of a function $f$ at $x=a$ (if it exists) is defined by the limit
\[
f'(a)=\lim_{h\to 0}\frac{f(a+h)-f(a)}{h}.
\]
It represents the \emph{instantaneous rate of change} of $f$ with respect to $x$ at $x=a$.

\medskip

\textbf{Geometric Meaning.}
If $y=f(x)$, then $f'(a)$ is the slope of the line tangent to the graph of $f$ at the point $(a,f(a))$.

\medskip

\textbf{Derivative Function.}
If the limit exists for each $x$ in an interval, then the derivative function is
\[
f'(x)=\lim_{h\to 0}\frac{f(x+h)-f(x)}{h}.
\]

\medskip

\textbf{Notation.}
If $y=f(x)$, the derivative may be written as
\[
f'(x),\qquad y',\qquad \frac{dy}{dx},\qquad \frac{df}{dx}.
\]
The notation $\frac{dy}{dx}$ is called \emph{Leibniz notation}; it reminds us the derivative measures how $y$ changes as $x$ changes.

\medskip

\textbf{Differentiable.}
A function is \emph{differentiable} at $x=a$ if $f'(a)$ exists. Differentiability implies continuity, but a continuous function may fail to be differentiable at a sharp corner, cusp, or vertical tangent.

\medskip

\textbf{Practical Approximation.}
When $h$ is very small,
\[
f'(a)\approx \frac{f(a+h)-f(a)}{h},
\]
which is the slope of a secant line over a tiny interval.

\subsection*{Solved Examples}

\textbf{Example 1: Derivative at a Point (Limit Definition)}

\textbf{Problem.}
Suppose the cost (in dollars) of producing $x$ units is
\[
C(x)=0.02x^2+5x+1000.
\]
Find $C'(50)$ using the limit definition and interpret the meaning.

\textbf{Solution.}

\textbf{Step 1.} Start with the definition:
\[
C'(50)=\lim_{h\to 0}\frac{C(50+h)-C(50)}{h}.
\]

\textbf{Step 2.} Compute $C(50+h)$:
\[
C(50+h)=0.02(50+h)^2+5(50+h)+1000.
\]
Expand $(50+h)^2=2500+100h+h^2$:
\[
C(50+h)=0.02(2500+100h+h^2)+250+5h+1000.
\]
\[
C(50+h)=50+2h+0.02h^2+250+5h+1000
=1300+7h+0.02h^2.
\]

\textbf{Step 3.} Compute $C(50)$:
\[
C(50)=0.02(2500)+5(50)+1000=50+250+1000=1300.
\]

\textbf{Step 4.} Form the difference quotient:
\[
\frac{C(50+h)-C(50)}{h}
=\frac{(1300+7h+0.02h^2)-1300}{h}
=\frac{7h+0.02h^2}{h}
=7+0.02h.
\]

\textbf{Step 5.} Take the limit:
\[
C'(50)=\lim_{h\to 0}\left(7+0.02h\right)=7.
\]

\textbf{Interpretation.}
At $x=50$ units, the cost is increasing at about \$7 per additional unit. In business language, \$7 is the \emph{marginal cost} at 50 units.

\medskip

\textbf{Example 2: Marginal Revenue from a Revenue Model}

\textbf{Problem.}
Revenue (in dollars) from selling $x$ units is
\[
R(x)=200x-0.5x^2.
\]
Use the limit definition to find $R'(x)$, then evaluate $R'(100)$ and interpret.

\textbf{Solution.}

\textbf{Step 1.} Use the definition:
\[
R'(x)=\lim_{h\to 0}\frac{R(x+h)-R(x)}{h}.
\]

\textbf{Step 2.} Compute $R(x+h)$:
\[
R(x+h)=200(x+h)-0.5(x+h)^2
=200x+200h-0.5(x^2+2xh+h^2).
\]
\[
R(x+h)=200x+200h-0.5x^2-xh-0.5h^2.
\]

\textbf{Step 3.} Subtract $R(x)=200x-0.5x^2$:
\[
R(x+h)-R(x)=\left(200x+200h-0.5x^2-xh-0.5h^2\right)-\left(200x-0.5x^2\right)
=200h-xh-0.5h^2.
\]

\textbf{Step 4.} Divide by $h$ and take the limit:
\[
\frac{R(x+h)-R(x)}{h}=200-x-0.5h
\quad\Rightarrow\quad
R'(x)=\lim_{h\to 0}(200-x-0.5h)=200-x.
\]

\textbf{Step 5.} Evaluate at $x=100$:
\[
R'(100)=200-100=100.
\]

\textbf{Interpretation.}
At 100 units, revenue is increasing at about \$100 per additional unit sold. This is the \emph{marginal revenue} at $x=100$.

\medskip

\textbf{Example 3: Approximating a Derivative from a Table}

\textbf{Problem.}
Let $S(t)$ be the total sales revenue (in thousands of dollars) after $t$ months. The table shows values of $S(t)$:

\[
\begin{array}{c|ccccc}
t & 1 & 2 & 3 & 4 & 5\\ \hline
S(t) & 42 & 47 & 55 & 66 & 80
\end{array}
\]

Approximate $S'(3)$ and interpret the result.

\textbf{Solution.}

\textbf{Step 1.} Use nearby average rates of change (secant slopes).

Using the interval from $t=2$ to $t=3$:
\[
\frac{S(3)-S(2)}{3-2}=\frac{55-47}{1}=8.
\]

Using the interval from $t=3$ to $t=4$:
\[
\frac{S(4)-S(3)}{4-3}=\frac{66-55}{1}=11.
\]

\textbf{Step 2.} Combine the information.
A reasonable estimate is to average these two:
\[
S'(3)\approx \frac{8+11}{2}=9.5.
\]

\textbf{Interpretation.}
At month 3, sales revenue is increasing at about $9.5$ thousand dollars per month (about \$9{,}500 per month).

\subsection*{Practice Problems}

\begin{enumerate}
  \item Let the cost (in dollars) be $C(x)=0.01x^2+4x+500$.
  \begin{enumerate}
    \item Use the limit definition to find $C'(x)$.
    
    \workbox[2.0in]{\textit{Hint: Start with }$\displaystyle C'(x)=\lim_{h\to 0}\frac{C(x+h)-C(x)}{h}.$}

    \item Evaluate $C'(80)$.
    
    \workbox[1.0in]{}

    \item Interpret $C'(80)$ in context.
    
    \workbox[1.4in]{}
  \end{enumerate}

  \begin{answer}
  \textbf{The Derivative, Problem 1.}
  \begin{enumerate}
    \item $C'(x)=0.02x+4$.
    \item $C'(80)=0.02(80)+4=5.6$.
    \item At 80 units, cost is increasing at about \$5.60 per additional unit (marginal cost).
  \end{enumerate}
  \end{answer}

  \item Revenue is modeled by $R(x)=150x-0.25x^2$ (dollars).
  \begin{enumerate}
    \item Find $R'(x)$ using the limit definition.
    
    \workbox[2.0in]{\textit{Hint: Expand } $(x+h)^2\textit{ and simplify before dividing by }h.$}

    \item Compute $R'(60)$.
    
    \workbox[1.0in]{}

    \item Interpret $R'(60)$ in context.
    
    \workbox[1.4in]{}
  \end{enumerate}

  \begin{answer}
  \textbf{The Derivative, Problem 2.}
  \begin{enumerate}
    \item $R'(x)=150-0.5x$.
    \item $R'(60)=150-30=120$.
    \item At 60 units, revenue is increasing at about \$120 per additional unit (marginal revenue).
  \end{enumerate}
  \end{answer}

  \item Total profit $P(t)$ (in thousands of dollars) after $t$ weeks is estimated by:
  \[
  \begin{array}{c|ccccc}
  t & 2 & 3 & 4 & 5 & 6\\ \hline
  P(t) & 18 & 22 & 27 & 31 & 34
  \end{array}
  \]
  \begin{enumerate}
    \item Approximate $P'(4)$ using secant slopes on both sides of $t=4$.
    
    \workbox[2.0in]{\textit{Hint: Use }$\frac{P(4)-P(3)}{4-3}\textit{ and }\frac{P(5)-P(4)}{5-4}\textit{, then average.}$}

    \item Interpret your result.
    
    \workbox[1.4in]{}
  \end{enumerate}

  \begin{answer}
  \textbf{The Derivative, Problem 3.}
  \begin{enumerate}
    \item Left slope: $(27-22)/1=5$. Right slope: $(31-27)/1=4$. Average: $(5+4)/2=4.5$.
    \item At week 4, profit is increasing at about \$4.5 thousand per week (about \$4{,}500 per week).
  \end{enumerate}
  \end{answer}
\end{enumerate}

\subsection*{Section Summary}

\begin{itemize}
  \item The derivative measures an instantaneous rate of change and equals the slope of a tangent line.
  \item The derivative at $x=a$ is defined by the limit $\displaystyle f'(a)=\lim_{h\to 0}\frac{f(a+h)-f(a)}{h}$.
  \item Derivatives can be interpreted as marginal quantities in business (marginal cost, marginal revenue, marginal profit).
  \item When only tables or graphs are available, derivatives can be approximated using nearby average rates of change.
  \item A function must be differentiable at a point for the derivative to exist there.
\end{itemize}


\newpage
\section{Power and Sum Rules for Derivatives}

\subsection*{Learning Objectives}
After completing this section, you should be able to:
\begin{itemize}
  \item Use the power rule to differentiate polynomial functions.
  \item Apply the constant multiple rule and sum rule for derivatives.
  \item Differentiate cost, revenue, and profit functions efficiently.
  \item Interpret derivatives obtained using rules as marginal quantities.
  \item Combine differentiation rules to handle realistic business models.
\end{itemize}

\subsection*{Key Definitions and Concepts}

\textbf{Power Rule.}
If $f(x)=x^n$, where $n$ is a real number, then the derivative of $f$ is
\[
\frac{d}{dx}\left(x^n\right)=nx^{\,n-1}.
\]

This rule allows derivatives of polynomial terms to be computed quickly without using limits.

\medskip

\textbf{Constant Multiple Rule.}
If $f(x)$ is differentiable and $c$ is a constant, then
\[
\frac{d}{dx}\bigl[c\,f(x)\bigr]=c\,f'(x).
\]

\medskip

\textbf{Sum and Difference Rules.}
If $f(x)$ and $g(x)$ are differentiable, then
\[
\frac{d}{dx}\bigl[f(x)\pm g(x)\bigr]=f'(x)\pm g'(x).
\]

\medskip

\textbf{Derivative of a Constant.}
If $f(x)=c$, where $c$ is a constant, then
\[
\frac{d}{dx}(c)=0.
\]

\medskip

\textbf{Business Interpretation.}
When cost, revenue, or profit functions are differentiated using these rules, the resulting derivative represents a \emph{marginal} quantity, such as marginal cost or marginal revenue.

\subsection*{Solved Examples}

\textbf{Example 1: Using the Power Rule}

\textbf{Problem.}
Find the derivative of
\[
f(x)=7x^4.
\]

\textbf{Solution.}

Using the power rule,
\[
f'(x)=7(4)x^{3}=28x^3.
\]

\medskip

\textbf{Example 2: Applying the Sum Rule}

\textbf{Problem.}
Find the derivative of
\[
g(x)=3x^3-5x^2+12x.
\]

\textbf{Solution.}

Differentiate each term separately:
\[
g'(x)=9x^2-10x+12.
\]

\medskip

\textbf{Example 3: Marginal Cost}

\textbf{Problem.}
Suppose the cost (in dollars) of producing $x$ units is
\[
C(x)=0.01x^3+4x^2+200x+500.
\]
Find the marginal cost function and evaluate it at $x=50$.

\textbf{Solution.}

\textbf{Step 1.} Differentiate using the power and sum rules:
\[
C'(x)=0.03x^2+8x+200.
\]

\textbf{Step 2.} Evaluate at $x=50$:
\[
C'(50)=0.03(2500)+8(50)+200=75+400+200=675.
\]

\textbf{Interpretation.}
At 50 units, the cost is increasing at about \$675 per additional unit.

\medskip

\textbf{Example 4: Marginal Profit}

\textbf{Problem.}
Profit (in dollars) from selling $x$ units is given by
\[
P(x)=-0.02x^2+40x-500.
\]
Find the marginal profit function.

\textbf{Solution.}

Differentiate term by term:
\[
P'(x)=-0.04x+40.
\]

\subsection*{Practice Problems}

\begin{enumerate}
  \item Find the derivative of
  \[
  f(x)=5x^6.
  \]
  \workbox[1.2in]{}

  \item Find the derivative of
  \[
  g(x)=4x^3-7x^2+9.
  \]
  \workbox[1.4in]{}

  \item The cost (in dollars) of producing $x$ units is
  \[
  C(x)=0.02x^3+6x^2+150x+800.
  \]
  \begin{enumerate}
    \item Find the marginal cost function.
    \workbox[1.6in]{}
    \item Find the marginal cost when $x=40$.
    \workbox[1.2in]{}
  \end{enumerate}

  \item Revenue (in dollars) from selling $x$ units is
  \[
  R(x)=120x-0.3x^2.
  \]
  \begin{enumerate}
    \item Find the marginal revenue function.
    \workbox[1.4in]{}
    \item Interpret $R'(50)$.
    \workbox[1.4in]{}
  \end{enumerate}

  \begin{answer}
  \textbf{Power and Sum Rules for Derivatives, Practice Problems.}
  \begin{enumerate}
    \item $f'(x)=30x^5$.
    \item $g'(x)=12x^2-14x$.
    \item
    \begin{enumerate}
      \item $C'(x)=0.06x^2+12x+150$.
      \item $C'(40)=0.06(1600)+480+150=726$.
    \end{enumerate}
    \item
    \begin{enumerate}
      \item $R'(x)=120-0.6x$.
      \item At 50 units, revenue is decreasing at about \$10 per additional unit.
    \end{enumerate}
  \end{enumerate}
  \end{answer}
\end{enumerate}

\subsection*{Section Summary}

\begin{itemize}
  \item The power rule provides a fast method for differentiating powers of $x$.
  \item Constants differentiate to zero, and constant multiples factor out of derivatives.
  \item The derivative of a sum or difference is the sum or difference of the derivatives.
  \item These rules simplify the process of finding marginal cost, revenue, and profit.
  \item Differentiation rules allow complex business models to be analyzed efficiently.
\end{itemize}

\newpage
\section{Product and Quotient Rules}

\subsection*{Learning Objectives}
After completing this section, you should be able to:
\begin{itemize}
  \item Use the product rule to differentiate products of functions.
  \item Use the quotient rule to differentiate ratios of functions.
  \item Apply product and quotient rules to business-related functions.
  \item Combine differentiation rules to handle realistic cost, revenue, and profit models.
  \item Interpret derivatives obtained using these rules as marginal quantities.
\end{itemize}

\subsection*{Key Definitions and Concepts}

\textbf{Product Rule.}
If $f(x)$ and $g(x)$ are differentiable functions, then the derivative of their product is
\[
\frac{d}{dx}\bigl[f(x)g(x)\bigr]
= f'(x)g(x)+f(x)g'(x).
\]

\medskip

\textbf{Quotient Rule.}
If $f(x)$ and $g(x)$ are differentiable and $g(x)\neq 0$, then
\[
\frac{d}{dx}\!\left(\frac{f(x)}{g(x)}\right)
= \frac{f'(x)g(x)-f(x)g'(x)}{[g(x)]^2}.
\]

\medskip

\textbf{When to Use These Rules.}
\begin{itemize}
  \item Use the product rule when a function is written as a \emph{product} of two functions.
  \item Use the quotient rule when a function is written as a \emph{ratio} of two functions.
  \item Do \emph{not} distribute derivatives across products or quotients.
\end{itemize}

\medskip

\textbf{Business Interpretation.}
Product and quotient rules often arise when costs, revenues, or profits depend on multiple interacting variables, such as price times quantity or averages per unit.

\subsection*{Solved Examples}

\textbf{Example 1: Using the Product Rule}

\textbf{Problem.}
Suppose revenue (in dollars) is given by
\[
R(x)=(50x)(100-2x),
\]
where $x$ is the number of units sold. Find $R'(x)$.

\textbf{Solution.}

Let
\[
f(x)=50x, \qquad g(x)=100-2x.
\]

\textbf{Step 1.} Compute the derivatives:
\[
f'(x)=50, \qquad g'(x)=-2.
\]

\textbf{Step 2.} Apply the product rule:
\[
R'(x)=f'(x)g(x)+f(x)g'(x)
=50(100-2x)+50x(-2).
\]

\textbf{Step 3.} Simplify:
\[
R'(x)=5000-100x-100x=5000-200x.
\]

\medskip

\textbf{Example 2: Interpreting Marginal Revenue}

\textbf{Problem.}
Using the result from Example 1, evaluate $R'(20)$ and interpret.

\textbf{Solution.}
\[
R'(20)=5000-200(20)=5000-4000=1000.
\]

\textbf{Interpretation.}
At 20 units sold, revenue is increasing at about \$1{,}000 per additional unit.

\medskip

\textbf{Example 3: Using the Quotient Rule}

\textbf{Problem.}
Suppose the average cost (in dollars per unit) of producing $x$ units is
\[
A(x)=\frac{0.02x^2+200x+1000}{x}.
\]
Find $A'(x)$.

\textbf{Solution.}

Let
\[
f(x)=0.02x^2+200x+1000, \qquad g(x)=x.
\]

\textbf{Step 1.} Compute derivatives:
\[
f'(x)=0.04x+200, \qquad g'(x)=1.
\]

\textbf{Step 2.} Apply the quotient rule:
\[
A'(x)=\frac{(0.04x+200)(x)-(0.02x^2+200x+1000)(1)}{x^2}.
\]

\textbf{Step 3.} Simplify the numerator:
\[
A'(x)=\frac{0.04x^2+200x-0.02x^2-200x-1000}{x^2}
=\frac{0.02x^2-1000}{x^2}.
\]

\textbf{Example 4: Simplifying Before Differentiation}

\textbf{Problem.}
Simplify $A(x)$ from Example 3 before differentiating, then find $A'(x)$.

\textbf{Solution.}

First simplify:
\[
A(x)=0.02x+200+\frac{1000}{x}.
\]

Differentiate term by term:
\[
A'(x)=0.02-\frac{1000}{x^2}.
\]

This result matches the simplified form of the quotient rule result.

\subsection*{Practice Problems}

\begin{enumerate}
  \item Find the derivative of
  \[
  f(x)=(4x)(x^2+3).
  \]
  \workbox[1.6in]{}

  \item Find the derivative of
  \[
  g(x)=\frac{6x^2+50x}{x}.
  \]
  \workbox[1.6in]{}

  \item Revenue (in dollars) is given by
  \[
  R(x)=x(120-3x).
  \]
  \begin{enumerate}
    \item Find $R'(x)$.
    \workbox[1.6in]{}
    \item Evaluate $R'(10)$ and interpret the result.
    \workbox[1.4in]{}
  \end{enumerate}

  \item The average cost (in dollars per unit) of producing $x$ units is
  \[
  A(x)=\frac{0.05x^2+300x+2000}{x}.
  \]
  \begin{enumerate}
    \item Find $A'(x)$.
    \workbox[1.8in]{}
    \item Determine whether average cost is increasing or decreasing at $x=50$.
    \workbox[1.4in]{}
  \end{enumerate}

  \begin{answer}
  \textbf{Product and Quotient Rules, Practice Problems.}
  \begin{enumerate}
    \item $f'(x)=4(x^2+3)+4x(2x)=12x^2+12$.
    \item $g'(x)=\frac{(12x+50)x-(6x^2+50x)}{x^2}=\frac{6x^2}{x^2}=6$.
    \item
    \begin{enumerate}
      \item $R'(x)=120-6x$.
      \item $R'(10)=60$; revenue is increasing at \$60 per additional unit.
    \end{enumerate}
    \item
    \begin{enumerate}
      \item $A'(x)=0.05-\frac{2000}{x^2}$.
      \item At $x=50$, $A'(50)=0.05-\frac{2000}{2500}=-0.75$, so average cost is decreasing.
    \end{enumerate}
  \end{enumerate}
  \end{answer}
\end{enumerate}

\subsection*{Section Summary}

\begin{itemize}
  \item The product rule is used to differentiate products of functions.
  \item The quotient rule is used to differentiate ratios of functions.
  \item Simplifying expressions before differentiating can reduce algebraic complexity.
  \item Product and quotient rules frequently arise in business and economic models.
  \item Derivatives obtained using these rules often represent marginal or average rates of change.
\end{itemize}

\newpage
\section{Chain Rule}

\subsection*{Learning Objectives}
After completing this section, you should be able to:
\begin{itemize}
  \item Recognize composite functions.
  \item Apply the chain rule to differentiate composite functions.
  \item Differentiate business functions involving powers and nested expressions.
  \item Combine the chain rule with power, product, and quotient rules.
  \item Interpret derivatives obtained using the chain rule in business contexts.
\end{itemize}

\subsection*{Key Definitions and Concepts}

\textbf{Composite Function.}
A function is called a \emph{composite function} if it can be written in the form
\[
y=f(g(x)),
\]
where $g(x)$ is an inner function and $f(u)$ is an outer function.

\medskip

\textbf{Chain Rule.}
If $y=f(g(x))$ and both $f$ and $g$ are differentiable, then
\[
\frac{dy}{dx}=f'(g(x))\cdot g'(x).
\]

\medskip

\textbf{Alternative Notation.}
If $y=f(u)$ and $u=g(x)$, then
\[
\frac{dy}{dx}=\frac{dy}{du}\cdot\frac{du}{dx}.
\]

\medskip

\textbf{When the Chain Rule Is Needed.}
The chain rule is used whenever a function involves a power or expression applied to another function, such as
\[
(x^2+5x+1)^4 \quad \text{or} \quad \sqrt{3x^2+10}.
\]

\medskip

\textbf{Business Interpretation.}
In business and economics, the chain rule allows us to compute marginal changes when one quantity depends on another, which in turn depends on the input variable.

\subsection*{Solved Examples}

\textbf{Example 1: Basic Use of the Chain Rule}

\textbf{Problem.}
Find the derivative of
\[
f(x)=(3x^2+5)^4.
\]

\textbf{Solution.}

Let $u=3x^2+5$, so that $f(x)=u^4$.

Differentiate:
\[
\frac{df}{du}=4u^3, \qquad \frac{du}{dx}=6x.
\]

Apply the chain rule:
\[
f'(x)=4(3x^2+5)^3(6x)=24x(3x^2+5)^3.
\]

\medskip

\textbf{Example 2: Chain Rule with a Power}

\textbf{Problem.}
Find the derivative of
\[
g(x)=\sqrt{5x^2+20}.
\]

\textbf{Solution.}

Rewrite the function:
\[
g(x)=(5x^2+20)^{1/2}.
\]

Differentiate using the chain rule:
\[
g'(x)=\frac{1}{2}(5x^2+20)^{-1/2}(10x)
=\frac{5x}{\sqrt{5x^2+20}}.
\]

\medskip

\textbf{Example 3: Marginal Cost with the Chain Rule}

\textbf{Problem.}
Suppose the cost (in dollars) of producing $x$ units is
\[
C(x)=(0.01x^2+4x+500)^2.
\]
Find the marginal cost function.

\textbf{Solution.}

Let $u=0.01x^2+4x+500$, so $C(x)=u^2$.

Differentiate:
\[
\frac{dC}{du}=2u, \qquad \frac{du}{dx}=0.02x+4.
\]

Apply the chain rule:
\[
C'(x)=2(0.01x^2+4x+500)(0.02x+4).
\]

\textbf{Interpretation.}
The marginal cost depends on both the cost level and how rapidly the cost level is changing.

\subsection*{Practice Problems}

\begin{enumerate}
  \item Find the derivative of
  \[
  f(x)=(x^2+3x+1)^5.
  \]
  \workbox[1.8in]{}

  \item Find the derivative of
  \[
  g(x)=\sqrt{2x^2+10x}.
  \]
  \workbox[1.6in]{}

  \item The revenue (in dollars) from selling $x$ units is
  \[
  R(x)=(100-2x)^3.
  \]
  \begin{enumerate}
    \item Find $R'(x)$.
    \workbox[1.8in]{}
    \item Interpret $R'(20)$.
    \workbox[1.4in]{}
  \end{enumerate}

  \item The profit (in dollars) from selling $x$ units is
  \[
  P(x)=(x^2+10x+25)^{1/2}.
  \]
  \begin{enumerate}
    \item Find $P'(x)$.
    \workbox[1.8in]{}
    \item Determine $P'(25)$.
    \workbox[1.2in]{}
  \end{enumerate}

  \begin{answer}
  \textbf{Chain Rule, Practice Problems.}
  \begin{enumerate}
    \item $f'(x)=5(x^2+3x+1)^4(2x+3)$.
    \item $g'(x)=\dfrac{2x+5}{\sqrt{2x^2+10x}}$.
    \item
    \begin{enumerate}
      \item $R'(x)=-6(100-2x)^2$.
      \item $R'(20)=-6(60)^2=-21{,}600$; revenue is decreasing at about \$21{,}600 per additional unit.
    \end{enumerate}
    \item
    \begin{enumerate}
      \item $P'(x)=\dfrac{x+5}{\sqrt{x^2+10x+25}}$.
      \item $P'(25)=\dfrac{30}{\sqrt{900}}=1$.
    \end{enumerate}
  \end{enumerate}
  \end{answer}
\end{enumerate}

\subsection*{Section Summary}

\begin{itemize}
  \item The chain rule is used to differentiate composite functions.
  \item It allows differentiation of powers applied to inner functions.
  \item The chain rule frequently appears in business models involving nested relationships.
  \item It is often combined with other differentiation rules.
  \item Chain rule derivatives often represent marginal changes in complex systems.
\end{itemize}


\newpage
\section{Second Derivative and Concavity}

\subsection*{Learning Objectives}
After completing this section, you should be able to:
\begin{itemize}
  \item Compute second derivatives of functions.
  \item Interpret the second derivative as a rate of change of a rate of change.
  \item Determine intervals where a function is concave up or concave down.
  \item Identify points of inflection using the second derivative.
  \item Apply concavity concepts to business and economic models.
\end{itemize}

\subsection*{Key Definitions and Concepts}

\textbf{Second Derivative.}
If $f(x)$ is a differentiable function and its derivative $f'(x)$ is also differentiable, then the \emph{second derivative} of $f$ is
\[
f''(x)=\frac{d}{dx}\bigl[f'(x)\bigr].
\]
The second derivative measures how the rate of change of $f$ itself is changing.

\medskip

\textbf{Interpretation.}
\begin{itemize}
  \item $f'(x)$ measures the rate of change of $f(x)$.
  \item $f''(x)$ measures how fast that rate of change is increasing or decreasing.
\end{itemize}

\medskip

\textbf{Concavity.}
\begin{itemize}
  \item A function is \emph{concave up} on an interval if $f''(x) > 0$ on that interval.
  \item A function is \emph{concave down} on an interval if $f''(x) < 0$ on that interval.
\end{itemize}

\medskip

\textbf{Point of Inflection.}
A point $x=c$ is called a \emph{point of inflection} if the concavity of $f(x)$ changes at $x=c$. This typically occurs where $f''(c)=0$ or $f''(c)$ is undefined and concavity changes.

\medskip

\textbf{Business Interpretation.}
In business and economics:
\begin{itemize}
  \item A positive second derivative may indicate increasing marginal cost or accelerating growth.
  \item A negative second derivative may indicate decreasing marginal profit or slowing growth.
\end{itemize}

\subsection*{Solved Examples}

\textbf{Example 1: Computing a Second Derivative}

\textbf{Problem.}
Let
\[
f(x)=2x^3-15x^2+36x+10.
\]
Find $f'(x)$ and $f''(x)$.

\textbf{Solution.}

Differentiate once:
\[
f'(x)=6x^2-30x+36.
\]

Differentiate again:
\[
f''(x)=12x-30.
\]

\medskip

\textbf{Example 2: Concavity of a Cost Function}

\textbf{Problem.}
Suppose the cost (in dollars) of producing $x$ units is
\[
C(x)=0.5x^3-6x^2+40x+100.
\]
Determine where the cost function is concave up or concave down.

\textbf{Solution.}

\textbf{Step 1.} Find the first derivative:
\[
C'(x)=1.5x^2-12x+40.
\]

\textbf{Step 2.} Find the second derivative:
\[
C''(x)=3x-12.
\]

\textbf{Step 3.} Determine where $C''(x)$ is positive or negative.

Set $C''(x)=0$:
\[
3x-12=0 \quad \Rightarrow \quad x=4.
\]

\textbf{Step 4.} Test intervals:
\begin{itemize}
  \item For $x<4$, $C''(x)<0$ (concave down).
  \item For $x>4$, $C''(x)>0$ (concave up).
\end{itemize}

\medskip

\textbf{Example 3: Point of Inflection}

\textbf{Problem.}
Let profit (in dollars) be given by
\[
P(x)=-x^3+12x^2-36x+20.
\]
Find the point of inflection.

\textbf{Solution.}

\textbf{Step 1.} Compute the derivatives:
\[
P'(x)=-3x^2+24x-36,
\]
\[
P''(x)=-6x+24.
\]

\textbf{Step 2.} Set the second derivative equal to zero:
\[
-6x+24=0 \quad \Rightarrow \quad x=4.
\]

\textbf{Step 3.} Verify concavity change:
\begin{itemize}
  \item For $x<4$, $P''(x)>0$ (concave up).
  \item For $x>4$, $P''(x)<0$ (concave down).
\end{itemize}

\textbf{Conclusion.}
The profit function has a point of inflection at $x=4$.

\subsection*{Practice Problems}

\begin{enumerate}
  \item Find the first and second derivatives of
  \[
  f(x)=x^4-8x^3+18x^2.
  \]
  \workbox[2.0in]{}

  \item The revenue (in dollars) from selling $x$ units is
  \[
  R(x)=3x^3-30x^2+90x.
  \]
  \begin{enumerate}
    \item Find $R'(x)$ and $R''(x)$.
    \workbox[2.0in]{}
    \item Determine where the revenue function is concave up or concave down.
    \workbox[1.6in]{}
  \end{enumerate}

  \item The cost (in dollars) of producing $x$ units is
  \[
  C(x)=0.25x^3-6x^2+48x+200.
  \]
  \begin{enumerate}
    \item Find $C''(x)$.
    \workbox[1.2in]{}
    \item Find any point(s) of inflection.
    \workbox[1.6in]{}
  \end{enumerate}

  \begin{answer}
  \textbf{Second Derivative and Concavity, Practice Problems.}
  \begin{enumerate}
    \item $f'(x)=4x^3-24x^2+36x$, \quad $f''(x)=12x^2-48x+36$.
    \item
    \begin{enumerate}
      \item $R'(x)=9x^2-60x+90$, \quad $R''(x)=18x-60$.
      \item $R''(x)=0$ at $x=\tfrac{10}{3}$. Concave down for $x<\tfrac{10}{3}$, concave up for $x>\tfrac{10}{3}$.
    \end{enumerate}
    \item
    \begin{enumerate}
      \item $C''(x)=1.5x-12$.
      \item $C''(x)=0$ at $x=8$, which is a point of inflection.
    \end{enumerate}
  \end{enumerate}
  \end{answer}
\end{enumerate}

\subsection*{Section Summary}

\begin{itemize}
  \item The second derivative measures how a rate of change is itself changing.
  \item Concavity describes the overall shape of a graph.
  \item A positive second derivative indicates concave up behavior.
  \item A negative second derivative indicates concave down behavior.
  \item Points of inflection occur where concavity changes.
\end{itemize}

\newpage
\section{Optimization}

\subsection*{Learning Objectives}
After completing this section, you should be able to:
\begin{itemize}
  \item Explain the goal of optimization problems in business and economics.
  \item Use derivatives to find critical points of a function.
  \item Apply the first derivative test to identify maximum and minimum values.
  \item Solve optimization problems involving cost, revenue, and profit.
  \item Interpret optimal solutions in real-world business contexts.
\end{itemize}

\subsection*{Key Definitions and Concepts}

\textbf{Optimization.}
Optimization involves finding the \emph{maximum} or \emph{minimum} value of a function subject to given constraints.

\medskip

\textbf{Critical Point.}
A critical point of a function $f(x)$ is a value $x=c$ where
\[
f'(c)=0 \quad \text{or} \quad f'(c) \text{ is undefined}.
\]

\medskip

\textbf{Absolute vs.\ Relative Extrema.}
\begin{itemize}
  \item A \emph{relative (local) maximum or minimum} occurs near a point.
  \item An \emph{absolute (global) maximum or minimum} is the highest or lowest value over an entire interval.
\end{itemize}

\medskip

\textbf{First Derivative Test.}
If $f'(x)$ changes sign at a critical point $c$:
\begin{itemize}
  \item from positive to negative, $f$ has a local maximum at $c$;
  \item from negative to positive, $f$ has a local minimum at $c$.
\end{itemize}

\medskip

\textbf{Business Interpretation.}
Optimization techniques are used to determine optimal production levels, pricing strategies, and resource allocation that maximize profit or minimize cost.

\subsection*{Solved Examples}

\textbf{Example 1: Maximizing Profit}

\textbf{Problem.}
Suppose profit (in dollars) from selling $x$ units is
\[
P(x)=-2x^2+200x-1000.
\]
Find the production level that maximizes profit and determine the maximum profit.

\textbf{Solution.}

\textbf{Step 1.} Compute the first derivative:
\[
P'(x)=-4x+200.
\]

\textbf{Step 2.} Find critical points by setting $P'(x)=0$:
\[
-4x+200=0 \quad \Rightarrow \quad x=50.
\]

\textbf{Step 3.} Use the first derivative test.

For $x<50$, $P'(x)>0$ (profit increasing).  
For $x>50$, $P'(x)<0$ (profit decreasing).

\textbf{Conclusion.}
Profit is maximized at $x=50$ units.

\textbf{Step 4.} Find the maximum profit:
\[
P(50)=-2(50)^2+200(50)-1000=4000.
\]

\medskip

\textbf{Example 2: Minimizing Average Cost}

\textbf{Problem.}
The average cost (in dollars per unit) of producing $x$ units is
\[
A(x)=0.5x+\frac{800}{x}.
\]
Find the production level that minimizes average cost.

\textbf{Solution.}

\textbf{Step 1.} Compute the derivative:
\[
A'(x)=0.5-\frac{800}{x^2}.
\]

\textbf{Step 2.} Set $A'(x)=0$:
\[
0.5-\frac{800}{x^2}=0
\quad \Rightarrow \quad
x^2=1600
\quad \Rightarrow \quad
x=40.
\]

\textbf{Step 3.} Interpret.
Average cost is minimized when 40 units are produced.

\medskip

\textbf{Example 3: Optimization with Constraints}

\textbf{Problem.}
A company can sell its product for \$60 per unit. The cost of producing $x$ units is
\[
C(x)=0.5x^2+20x+500.
\]
Find the production level that maximizes profit.

\textbf{Solution.}

\textbf{Step 1.} Write the revenue function:
\[
R(x)=60x.
\]

\textbf{Step 2.} Write the profit function:
\[
P(x)=R(x)-C(x)=-0.5x^2+40x-500.
\]

\textbf{Step 3.} Differentiate:
\[
P'(x)=-x+40.
\]

\textbf{Step 4.} Find the critical point:
\[
-x+40=0 \quad \Rightarrow \quad x=40.
\]

\textbf{Conclusion.}
Profit is maximized when 40 units are produced.

\subsection*{Practice Problems}

\begin{enumerate}
  \item Profit (in dollars) is given by
  \[
  P(x)=-x^2+120x-900.
  \]
  \begin{enumerate}
    \item Find the production level that maximizes profit.
    \workbox[1.4in]{}
    \item Find the maximum profit.
    \workbox[1.4in]{}
  \end{enumerate}

  \item The average cost (in dollars per unit) is
  \[
  A(x)=x+\frac{1000}{x}.
  \]
  \begin{enumerate}
    \item Find the value of $x$ that minimizes average cost.
    \workbox[1.4in]{}
    \item Interpret the result.
    \workbox[1.4in]{}
  \end{enumerate}

  \item Revenue (in dollars) from selling $x$ units is
  \[
  R(x)=-3x^2+180x.
  \]
  \begin{enumerate}
    \item Find the value of $x$ that maximizes revenue.
    \workbox[1.4in]{}
    \item Find the maximum revenue.
    \workbox[1.4in]{}
  \end{enumerate}

  \item A firm produces a product with cost
  \[
  C(x)=x^2+40x+600.
  \]
  The product sells for \$80 per unit.
  \begin{enumerate}
    \item Write the profit function.
    \workbox[1.6in]{}
    \item Find the production level that maximizes profit.
    \workbox[1.4in]{}
  \end{enumerate}

  \begin{answer}
  \textbf{Optimization, Practice Problems.}
  \begin{enumerate}
    \item
    \begin{enumerate}
      \item $P'(x)=-2x+120=0 \Rightarrow x=60$.
      \item $P(60)=2700$.
    \end{enumerate}
    \item
    \begin{enumerate}
      \item $A'(x)=1-\dfrac{1000}{x^2}=0 \Rightarrow x=\sqrt{1000}\approx31.6$.
      \item Average cost is minimized at about 32 units.
    \end{enumerate}
    \item
    \begin{enumerate}
      \item $R'(x)=-6x+180=0 \Rightarrow x=30$.
      \item $R(30)=2700$.
    \end{enumerate}
    \item
    \begin{enumerate}
      \item $P(x)=80x-(x^2+40x+600)=-x^2+40x-600$.
      \item $P'(x)=-2x+40=0 \Rightarrow x=20$.
    \end{enumerate}
  \end{enumerate}
  \end{answer}
\end{enumerate}

\subsection*{Section Summary}

\begin{itemize}
  \item Optimization problems involve finding maximum or minimum values.
  \item Critical points occur where the derivative is zero or undefined.
  \item The first derivative test helps classify extrema.
  \item Optimization is widely used in pricing, production, and cost analysis.
  \item Calculus provides a systematic approach to decision-making in business.
\end{itemize}

\newpage
\section{Curve Sketching}

\subsection*{Learning Objectives}
After completing this section, you should be able to:
\begin{itemize}
  \item Identify key features of a function needed for sketching its graph.
  \item Use first derivatives to determine intervals of increase and decrease.
  \item Use second derivatives to analyze concavity.
  \item Identify critical points and points of inflection.
  \item Sketch graphs of business-related functions using calculus-based analysis.
\end{itemize}

\subsection*{Key Definitions and Concepts}

\textbf{Curve Sketching.}
Curve sketching is the process of drawing a graph of a function by analyzing its key characteristics rather than plotting many individual points.

\medskip

\textbf{Critical Points.}
A critical point occurs at a value $x=c$ where
\[
f'(c)=0 \quad \text{or} \quad f'(c) \text{ is undefined}.
\]
Critical points are candidates for local maxima or minima.

\medskip

\textbf{Increasing and Decreasing Intervals.}
\begin{itemize}
  \item If $f'(x)>0$ on an interval, then $f$ is increasing on that interval.
  \item If $f'(x)<0$ on an interval, then $f$ is decreasing on that interval.
\end{itemize}

\medskip

\textbf{Concavity.}
\begin{itemize}
  \item If $f''(x)>0$, the graph is concave up.
  \item If $f''(x)<0$, the graph is concave down.
\end{itemize}

\medskip

\textbf{Point of Inflection.}
A point of inflection occurs where the concavity of a function changes, typically where
\[
f''(x)=0
\]
and the concavity changes sign.

\medskip

\textbf{Business Interpretation.}
Curve sketching helps visualize how cost, revenue, or profit behaves over time or levels of production, revealing trends such as increasing returns, diminishing returns, or saturation.

\subsection*{Solved Examples}

\textbf{Example 1: Curve Sketching a Profit Function}

\textbf{Problem.}
Suppose profit (in dollars) is given by
\[
P(x)=-x^3+12x^2-36x+20.
\]
Use calculus to analyze the graph of $P(x)$.

\textbf{Solution.}

\textbf{Step 1.} Compute the first derivative:
\[
P'(x)=-3x^2+24x-36.
\]

\textbf{Step 2.} Find critical points by setting $P'(x)=0$:
\[
-3x^2+24x-36=0
\quad\Rightarrow\quad
x^2-8x+12=0
\quad\Rightarrow\quad
x=2,\;6.
\]

\textbf{Step 3.} Determine increasing/decreasing intervals.

Test intervals:
\begin{itemize}
  \item For $x<2$, $P'(x)<0$ (decreasing).
  \item For $2<x<6$, $P'(x)>0$ (increasing).
  \item For $x>6$, $P'(x)<0$ (decreasing).
\end{itemize}

Thus, $x=2$ is a local minimum and $x=6$ is a local maximum.

\textbf{Step 4.} Compute the second derivative:
\[
P''(x)=-6x+24.
\]

\textbf{Step 5.} Find points of inflection:
\[
-6x+24=0 \Rightarrow x=4.
\]

\textbf{Step 6.} Determine concavity:
\begin{itemize}
  \item For $x<4$, $P''(x)>0$ (concave up).
  \item For $x>4$, $P''(x)<0$ (concave down).
\end{itemize}

\textbf{Conclusion.}
The profit function decreases, then increases, then decreases, with a point of inflection at $x=4$.

\medskip

\textbf{Example 2: Curve Sketching a Cost Function}

\textbf{Problem.}
Let the cost (in dollars) of producing $x$ units be
\[
C(x)=0.25x^3-3x^2+12x+100.
\]
Analyze the graph of $C(x)$.

\textbf{Solution.}

\textbf{Step 1.} First derivative:
\[
C'(x)=0.75x^2-6x+12.
\]

\textbf{Step 2.} Critical points:
\[
0.75x^2-6x+12=0
\quad\Rightarrow\quad
x^2-8x+16=0
\quad\Rightarrow\quad
x=4.
\]

\textbf{Step 3.} Second derivative:
\[
C''(x)=1.5x-6.
\]

\textbf{Step 4.} Inflection point:
\[
1.5x-6=0 \Rightarrow x=4.
\]

\textbf{Interpretation.}
The cost function changes concavity at $x=4$, indicating a change in how marginal cost is increasing.

\subsection*{Practice Problems}

\begin{enumerate}
  \item Let
  \[
  f(x)=x^3-9x^2+24x.
  \]
  \begin{enumerate}
    \item Find all critical points.
    \workbox[1.6in]{}
    \item Determine intervals of increase and decrease.
    \workbox[1.8in]{}
    \item Find any points of inflection.
    \workbox[1.6in]{}
  \end{enumerate}

  \item The revenue (in dollars) from selling $x$ units is
\[
R(x)=-x^3+15x^2-50x.
\]
\begin{enumerate}
  \item Find all critical points of $R(x)$.
  \workbox[1.8in]{\textit{Note: Critical points are not necessarily integers.}}

  \item Determine the intervals on which revenue is increasing and decreasing.
  \workbox[2.0in]{}

  \item Identify the intervals of concavity.
  \workbox[1.8in]{}
\end{enumerate}

  \begin{answer}
\textbf{Curve Sketching, Practice Problems.}
\begin{enumerate}
  \item
  \begin{enumerate}
    \item $f'(x)=3x^2-18x+24=3(x-2)(x-4)$, so critical points are $x=2$ and $x=4$.
    \item Increasing on $(2,4)$; decreasing on $(-\infty,2)$ and $(4,\infty)$.
    \item $f''(x)=6x-18$, so $f''(x)=0$ at $x=3$ (inflection point at $x=3$).
  \end{enumerate}

  \item
  \begin{enumerate}
    \item $R'(x)=-3x^2+30x-50=0 \Rightarrow x=5\pm \dfrac{5\sqrt{3}}{3}$.
    \item Revenue is increasing on $\left(5-\dfrac{5\sqrt{3}}{3},\,5+\dfrac{5\sqrt{3}}{3}\right)$ and decreasing outside this interval.
    \item $R''(x)=-6x+30$. Concave up for $x<5$ and concave down for $x>5$.
  \end{enumerate}

  \item
  \begin{enumerate}
    \item $C'(x)=4x^3-12x^2+12x=4x(x-1)(x-3)$, \quad
          $C''(x)=12x^2-24x+12=12(x-1)^2$.
    \item Since $C''(x)=12(x-1)^2\ge 0$ for all $x$, $C$ is concave up for all $x$ (no inflection point).
  \end{enumerate}
\end{enumerate}
\end{answer}

\end{enumerate}

\subsection*{Section Summary}

\begin{itemize}
  \item Curve sketching uses derivatives to analyze the shape of a graph.
  \item First derivatives determine increasing and decreasing behavior.
  \item Second derivatives determine concavity and inflection points.
  \item Curve sketching provides a visual understanding of business models.
  \item Calculus-based sketches reveal trends not obvious from formulas alone.
\end{itemize}


\newpage
\section{Applied Optimization}

\subsection*{Learning Objectives}
After completing this section, you should be able to:
\begin{itemize}
  \item Translate real-world business problems into optimization models.
  \item Identify objective functions and constraints.
  \item Use derivatives to solve applied optimization problems.
  \item Interpret optimal solutions in realistic business contexts.
  \item Distinguish between mathematical solutions and practical feasibility.
\end{itemize}

\subsection*{Key Definitions and Concepts}

\textbf{Objective Function.}  
The objective function represents the quantity to be maximized or minimized, such as profit, revenue, cost, or output.

\medskip

\textbf{Constraint.}  
A constraint is an equation that limits the possible values of the variables in a problem.

\medskip

\textbf{Applied Optimization Process.}
\begin{enumerate}
  \item Define variables clearly.
  \item Write the objective function.
  \item Use the constraint to express the objective function in one variable.
  \item Differentiate and find critical points.
  \item Interpret the solution in the business context.
\end{enumerate}

\medskip

\textbf{Business Interpretation.}  
Applied optimization helps firms allocate resources efficiently and make informed strategic decisions.

\subsection*{Solved Examples}

\textbf{Example 1: Maximizing Profit with a Linear Constraint}

\textbf{Problem.}  
A company sells two products. Product A yields a profit of \$30 per unit and Product B yields a profit of \$20 per unit. The company can produce at most 100 units total. How many units of each product should be produced to maximize profit?

\textbf{Solution.}

Let $x$ be the number of units of Product A.  
Then $100-x$ units of Product B are produced.

\textbf{Step 1.} Write the profit function:
\[
P(x)=30x+20(100-x)=10x+2000.
\]

\textbf{Step 2.} Analyze the function.

Since $P(x)$ is increasing for all $x$, profit is maximized at the largest feasible value of $x$.

\textbf{Conclusion.}  
Produce 100 units of Product A and 0 units of Product B.

\medskip

\textbf{Example 2: Minimizing Cost with a Packaging Constraint}

\textbf{Problem.}  
A company wants to design an open-top rectangular box with a square base that holds 500 cubic inches. Find the dimensions that minimize the amount of material used.

\textbf{Solution.}

Let $x$ be the side length of the square base and $h$ the height.

\textbf{Step 1.} Write the volume constraint:
\[
x^2h=500 \quad \Rightarrow \quad h=\frac{500}{x^2}.
\]

\textbf{Step 2.} Write the surface area function:
\[
S=x^2+4xh.
\]

Substitute for $h$:
\[
S(x)=x^2+\frac{2000}{x}.
\]

\textbf{Step 3.} Differentiate:
\[
S'(x)=2x-\frac{2000}{x^2}.
\]

\textbf{Step 4.} Find the critical point:
\[
2x=\frac{2000}{x^2}
\Rightarrow x^3=1000
\Rightarrow x=10.
\]

\textbf{Step 5.} Find the height:
\[
h=\frac{500}{10^2}=5.
\]

\textbf{Conclusion.}  
The optimal box has base $10\times10$ inches and height 5 inches.

\subsection*{Practice Problems}

\begin{enumerate}
  \item A company produces two products. Product X yields \$40 profit per unit and Product Y yields \$25 profit per unit. The company can produce at most 120 units total.
  \begin{enumerate}
    \item Write the profit function.
    \workbox[1.8in]{}
    \item Determine the production plan that maximizes profit.
    \workbox[1.6in]{}
  \end{enumerate}

  \item A rectangular storage area is to be built along a straight wall, using fencing on only three sides. The total fencing available is 200 meters.
  \begin{enumerate}
    \item Express the area as a function of one variable.
    \workbox[2.0in]{}
    \item Find the dimensions that maximize the area.
    \workbox[1.8in]{}
  \end{enumerate}

  \item A product has a demand function
  \[
  q = 500 - 5p,
  \]
  where $p$ is the price (in dollars) and $q$ is the number of units sold.
  \begin{enumerate}
    \item Write the revenue function in terms of $p$.
    \workbox[1.6in]{}
    \item Find the price that maximizes revenue.
    \workbox[1.6in]{}
  \end{enumerate}

  \begin{answer}
  \textbf{Applied Optimization, Practice Problems.}
  \begin{enumerate}
    \item
    \begin{enumerate}
      \item Let $x$ be units of Product X. Then
      \[
      P(x)=40x+25(120-x)=15x+3000.
      \]
      \item Profit is maximized at $x=120$ (all Product X).
    \end{enumerate}

    \item
    \begin{enumerate}
      \item Let $x$ be the width and $y$ the length:
      \[
      2x+y=200 \Rightarrow y=200-2x,
      \quad A(x)=x(200-2x).
      \]
      \item $A'(x)=200-4x=0 \Rightarrow x=50,\; y=100$.
    \end{enumerate}

    \item
    \begin{enumerate}
      \item $R(p)=pq=p(500-5p)=500p-5p^2$.
      \item $R'(p)=500-10p=0 \Rightarrow p=50$.
    \end{enumerate}
  \end{enumerate}
  \end{answer}
\end{enumerate}

\subsection*{Section Summary}

\begin{itemize}
  \item Applied optimization models real business decisions.
  \item Constraints reduce problems to single-variable functions.
  \item Derivatives identify optimal values.
  \item Interpretation ensures solutions are meaningful in practice.
  \item Optimization supports efficient use of resources and pricing strategies.
\end{itemize}



\newpage
\section{Other Applications}

\subsection*{Learning Objectives}
After completing this section, you should be able to:
\begin{itemize}
  \item Use tangent line approximation to estimate function values.
  \item Interpret tangent line approximation in business contexts.
  \item Compute and interpret elasticity of demand.
  \item Classify demand as elastic, inelastic, or unitary.
  \item Explain how elasticity affects revenue decisions.
\end{itemize}

\subsection*{Key Definitions and Concepts}

\textbf{Tangent Line Approximation (TLA).}  
Tangent line approximation uses the tangent line to a function at a known point to estimate the value of the function at a nearby point.

\medskip

To approximate $f(x)$ using TLA:
\begin{enumerate}
  \item Choose a value $a$ such that $a$ is close to $x$.
  \item The exact values of $f(a)$ and $f'(a)$ are known.
\end{enumerate}

The tangent line approximation formula is
\[
f(x) \approx f(a) + f'(a)(x-a).
\]

\medskip

Another way to express the same idea is
\[
\Delta y \approx f'(a)\Delta x.
\]

The accuracy of the approximation depends on how close $x$ is to $a$ and on the shape of the graph of $f$.

\medskip

\textbf{Elasticity of Demand.}  
If a demand function gives quantity $q$ in terms of price $p$, the elasticity of demand is defined by
\[
E = \left| \frac{p}{q}\frac{dq}{dp} \right|.
\]

Since demand typically decreases as price increases, $\frac{dq}{dp}$ is negative. The absolute value ensures elasticity is positive.

\medskip

\textbf{Classification of Demand.}
\begin{itemize}
  \item If $E < 1$, demand is \emph{inelastic}.
  \item If $E > 1$, demand is \emph{elastic}.
  \item If $E = 1$, demand is \emph{unitary}.
\end{itemize}

\medskip

\textbf{Interpretation of Elasticity.}  
If the price increases by 1\%, the demand will decrease by approximately $E$\%.

\subsection*{Solved Examples}

\textbf{Example 1: Tangent Line Approximation}

\textbf{Problem.}  
Let
\[
f(x)=\sqrt{x}.
\]
Use tangent line approximation at $x=16$ to estimate $\sqrt{15}$.

\textbf{Solution.}

\textbf{Step 1.} Compute $f(16)$ and $f'(x)$.
\[
f(16)=4, \quad f'(x)=\frac{1}{2\sqrt{x}}, \quad f'(16)=\frac{1}{8}.
\]

\textbf{Step 2.} Apply the approximation formula.
\[
f(15) \approx f(16) + f'(16)(15-16)
= 4 - \frac{1}{8}
= 3.875.
\]

\textbf{Interpretation.}  
$\sqrt{15}$ is approximately $3.875$.

\medskip

\textbf{Example 2: Computing Elasticity of Demand}

\textbf{Problem.}  
Suppose the demand function is
\[
q = 200 - 4p.
\]
Find the elasticity of demand when $p=20$ and classify the demand.

\textbf{Solution.}

\textbf{Step 1.} Compute $\dfrac{dq}{dp}$.
\[
\frac{dq}{dp} = -4.
\]

\textbf{Step 2.} Evaluate $q$ at $p=20$.
\[
q = 200 - 4(20) = 120.
\]

\textbf{Step 3.} Compute elasticity.
\[
E = \left| \frac{20}{120}(-4) \right| = \frac{2}{3}.
\]

\textbf{Conclusion.}  
Since $E<1$, demand is inelastic. Increasing price will increase revenue.

\subsection*{Practice Problems}

\begin{enumerate}
  \item Use tangent line approximation to estimate $\sqrt{24}$ using $a=25$.
  \workbox[2.0in]{}

  \item A demand function is given by
  \[
  q = 300 - 6p.
  \]
  \begin{enumerate}
    \item Find the elasticity of demand.
    \workbox[1.8in]{}
    \item Evaluate elasticity at $p=30$.
    \workbox[1.6in]{}
    \item Classify the demand.
    \workbox[1.4in]{}
  \end{enumerate}

  \item Explain what it means if demand is unitary.
  \workbox[1.6in]{}

  \begin{answer}
  \textbf{Other Applications, Practice Problems.}
  \begin{enumerate}
    \item $\sqrt{24}\approx 5 + \frac{1}{10}(24-25)=4.9$.
    \item
    \begin{enumerate}
      \item $E=\left|\dfrac{p}{q}\dfrac{dq}{dp}\right|$ with $\dfrac{dq}{dp}=-6$.
      \item At $p=30$, $q=120$ so $E=\left|\dfrac{30}{120}(-6)\right|=1.5$.
      \item Demand is elastic.
    \end{enumerate}
    \item When demand is unitary, a price change does not change total revenue.
  \end{enumerate}
  \end{answer}
\end{enumerate}

\subsection*{Section Summary}

\begin{itemize}
  \item Tangent line approximation estimates function values near known points.
  \item The derivative provides the linear rate of change.
  \item Elasticity measures responsiveness of demand to price changes.
  \item Elasticity guides pricing and revenue decisions.
  \item Calculus connects mathematical models to real business behavior.
\end{itemize}


\newpage
\section{Implicit Differentiation and Related Rates}

\subsection*{Learning Objectives}
After completing this section, you should be able to:
\begin{itemize}
  \item Differentiate equations defined implicitly.
  \item Solve for $\dfrac{dy}{dx}$ using implicit differentiation.
  \item Apply related rates to business and applied problems.
  \item Interpret rates of change in context.
  \item Translate word problems into mathematical relationships.
\end{itemize}

\subsection*{Key Definitions and Concepts}

\textbf{Implicit Differentiation.}  
Implicit differentiation is used when a function is not explicitly solved for one variable in terms of another. Instead of rewriting the equation, both sides are differentiated with respect to $x$.

\medskip

When differentiating terms involving $y$, the chain rule is applied:
\[
\frac{d}{dx}(y^n)=n y^{\,n-1}\frac{dy}{dx}.
\]

\medskip

\textbf{Related Rates.}  
Related rates problems involve finding the rate of change of one quantity by relating it to other changing quantities using derivatives.

\medskip

\textbf{General Strategy for Related Rates.}
\begin{enumerate}
  \item Write an equation relating the variables.
  \item Differentiate both sides with respect to time.
  \item Substitute known values.
  \item Solve for the desired rate.
\end{enumerate}

\medskip

\textbf{Business Interpretation.}  
Related rates help model how changes in production, revenue, cost, or resources affect one another over time.

\subsection*{Solved Examples}

\textbf{Example 1: Implicit Differentiation}

\textbf{Problem.}  
Differentiate the equation
\[
x^2 + xy + y^2 = 25
\]
with respect to $x$.

\textbf{Solution.}

Differentiate both sides with respect to $x$:
\[
2x + (x\,\frac{dy}{dx} + y) + 2y\frac{dy}{dx} = 0.
\]

Group terms involving $\dfrac{dy}{dx}$:
\[
(x + 2y)\frac{dy}{dx} = -(2x + y).
\]

Solve for $\dfrac{dy}{dx}$:
\[
\frac{dy}{dx} = -\frac{2x + y}{x + 2y}.
\]

\medskip

\textbf{Example 2: Related Rates in Production}

\textbf{Problem.}  
A company produces square metal sheets. Let $x$ be the length of a side (in meters) and $A$ the area (in square meters). If the side length is increasing at a rate of $0.5$ m/min, find the rate at which the area is increasing when $x=10$.

\textbf{Solution.}

\textbf{Step 1.} Write the equation:
\[
A=x^2.
\]

\textbf{Step 2.} Differentiate with respect to time $t$:
\[
\frac{dA}{dt}=2x\frac{dx}{dt}.
\]

\textbf{Step 3.} Substitute values:
\[
\frac{dA}{dt}=2(10)(0.5)=10.
\]

\textbf{Conclusion.}  
The area is increasing at $10$ m$^2$/min.

\medskip

\textbf{Example 3: Related Rates in Revenue}

\textbf{Problem.}  
Revenue is given by
\[
R=pq,
\]
where $p$ is price and $q$ is quantity sold. Suppose price is decreasing at \$2 per unit per month and sales volume is increasing at 50 units per month. Find the rate of change of revenue when $p=40$ and $q=500$.

\textbf{Solution.}

Differentiate with respect to time:
\[
\frac{dR}{dt}=p\frac{dq}{dt}+q\frac{dp}{dt}.
\]

Substitute values:
\[
\frac{dR}{dt}=40(50)+500(-2)=2000-1000=1000.
\]

\textbf{Interpretation.}  
Revenue is increasing at \$1{,}000 per month.

\subsection*{Practice Problems}

\begin{enumerate}
  \item Use implicit differentiation to find $\dfrac{dy}{dx}$:
  \[
  x^2 + y^2 + 4xy = 16.
  \]
  \workbox[2.2in]{}

  \item Suppose the area of a rectangular warehouse is
  \[
  A=lw,
  \]
  where $l$ and $w$ are changing over time. If $l$ is increasing at 3 m/month and $w$ is decreasing at 1 m/month, find $\dfrac{dA}{dt}$ when $l=20$ and $w=10$.
  \workbox[2.2in]{}

  \item A company’s inventory value is given by
  \[
  V=pq.
  \]
  If price is increasing at \$1 per unit per month and quantity is decreasing at 30 units per month, find $\dfrac{dV}{dt}$ when $p=25$ and $q=400$.
  \workbox[2.2in]{}

  \begin{answer}
  \textbf{Implicit Differentiation and Related Rates, Practice Problems.}
  \begin{enumerate}
    \item $\dfrac{dy}{dx}=-\dfrac{x+2y}{y+2x}$.
    \item $\dfrac{dA}{dt}=20(-1)+10(3)=10$ m$^2$/month.
    \item $\dfrac{dV}{dt}=25(-30)+400(1)=-350$ dollars per month.
  \end{enumerate}
  \end{answer}
\end{enumerate}

\subsection*{Section Summary}

\begin{itemize}
  \item Implicit differentiation is used when variables are interdependent.
  \item The chain rule accounts for variables changing with respect to another.
  \item Related rates problems involve differentiating with respect to time.
  \item Business applications include production, revenue, and inventory changes.
  \item Calculus connects multiple changing quantities in real-world systems.
\end{itemize}

