\chapter{Review}
\section{Functions and Graphs}

\subsection*{Learning Objectives}
After completing this section, you should be able to:
\begin{itemize}
  \item Understand and use function notation.
  \item Evaluate functions at given input values.
  \item Identify domains and ranges of basic functions.
  \item Interpret graphs of functions in business contexts.
  \item Recognize increasing and decreasing behavior from graphs and formulas.
\end{itemize}

\subsection*{Key Definitions and Concepts}

\textbf{Function.}  
A function is a rule that assigns exactly one output to each input.  
If $f$ is a function, we write
\[
y=f(x).
\]

\medskip

\textbf{Function Notation.}  
The symbol $f(x)$ represents the value of the function when the input is $x$.  
It does \emph{not} mean multiplication.

\medskip

\textbf{Domain.}  
The domain of a function is the set of all input values for which the function is defined.

\medskip

\textbf{Range.}  
The range of a function is the set of all possible output values.

\medskip

\textbf{Graph of a Function.}  
The graph of $f(x)$ is the set of all points $(x,f(x))$ in the coordinate plane.

\medskip

\textbf{Increasing and Decreasing Functions.}  
A function is increasing on an interval if larger inputs produce larger outputs, and decreasing if larger inputs produce smaller outputs.

\subsection*{Solved Examples}

\textbf{Example 1: Evaluating a Function}

\textbf{Problem.}  
Let
\[
f(x)=3x-5.
\]
Find $f(4)$ and interpret the result.

\textbf{Solution.}

\[
f(4)=3(4)-5=7.
\]

\textbf{Interpretation.}  
When the input is 4, the output of the function is 7.

\medskip

\textbf{Example 2: Domain of a Function}

\textbf{Problem.}  
Find the domain of
\[
g(x)=\frac{1}{x-2}.
\]

\textbf{Solution.}

The denominator cannot be zero, so $x\neq 2$.  
The domain is all real numbers except $x=2$.

\medskip

\textbf{Example 3: Interpreting a Graph}

\textbf{Problem.}  
Suppose $C(x)$ represents the cost (in dollars) of producing $x$ units of a product, and the graph of $C(x)$ is increasing. What does this mean?

\textbf{Solution.}

An increasing cost function means that producing more units leads to higher total cost.

\subsection*{Practice Problems}

\begin{enumerate}
  \item Let $f(x)=2x+1$.
  \begin{enumerate}
    \item Find $f(5)$.
    \workbox[1.2in]{}
    \item Find $x$ such that $f(x)=9$.
    \workbox[1.6in]{}
  \end{enumerate}

  \item Find the domain of the function
  \[
  h(x)=\frac{4}{x+3}.
  \]
  \workbox[1.8in]{}

  \item A revenue function $R(q)$ depends on the number of units sold, $q$.
  \begin{enumerate}
    \item What does $R(0)$ represent?
    \workbox[1.4in]{}
    \item If the graph of $R(q)$ is decreasing, what does this indicate?
    \workbox[1.8in]{}
  \end{enumerate}

  \item Describe the graph of the function $y=-3$.
  \workbox[1.6in]{}

  \begin{answer}
  \textbf{Functions and Graphs, Practice Problems.}
  \begin{enumerate}
    \item
    \begin{enumerate}
      \item $f(5)=11.$
      \item $x=4.$
    \end{enumerate}
    \item The domain is all real numbers except $x=-3$.
    \item
    \begin{enumerate}
      \item $R(0)$ represents the revenue when no units are sold.
      \item Revenue decreases as the number of units sold increases.
    \end{enumerate}
    \item A horizontal line 3 units below the $x$-axis.
  \end{enumerate}
  \end{answer}
\end{enumerate}

\subsection*{Section Summary}

\begin{itemize}
  \item Functions assign one output to each input.
  \item Function notation is used throughout calculus.
  \item Domains restrict allowable inputs.
  \item Graphs provide visual insight into function behavior.
  \item Understanding functions is essential for limits, derivatives, and integrals.
\end{itemize}

\newpage

\section{Algebraic Operations and Simplification}

\subsection*{Learning Objectives}
After completing this section, you should be able to:
\begin{itemize}
  \item Simplify algebraic expressions correctly.
  \item Factor basic polynomial expressions.
  \item Simplify rational expressions.
  \item Solve basic linear and quadratic equations.
  \item Apply algebraic simplification to prepare expressions for calculus.
\end{itemize}

\subsection*{Key Definitions and Concepts}

\textbf{Simplifying Expressions.}  
Simplifying an expression means rewriting it in an equivalent form that is easier to work with by combining like terms, factoring, or canceling common factors.

\medskip

\textbf{Factoring.}  
Factoring is the process of writing an expression as a product of simpler expressions.

\medskip

\textbf{Rational Expression.}  
A rational expression is a ratio of two polynomials. The expression is undefined when the denominator equals zero.

\medskip

\textbf{Why This Matters for Calculus.}  
Algebraic simplification is essential for:
\begin{itemize}
  \item evaluating limits,
  \item finding derivatives,
  \item computing integrals,
  \item avoiding unnecessary errors.
\end{itemize}

\subsection*{Solved Examples}

\textbf{Example 1: Simplifying an Expression}

\textbf{Problem.}  
Simplify:
\[
3x-2x+5.
\]

\textbf{Solution.}

\[
3x-2x+5=x+5.
\]

\medskip

\textbf{Example 2: Factoring}

\textbf{Problem.}  
Factor the expression
\[
x^2-9.
\]

\textbf{Solution.}

This is a difference of squares:
\[
x^2-9=(x-3)(x+3).
\]

\medskip

\textbf{Example 3: Simplifying a Rational Expression}

\textbf{Problem.}  
Simplify
\[
\frac{x^2-4}{x-2}, \quad x\neq 2.
\]

\textbf{Solution.}

Factor the numerator:
\[
\frac{(x-2)(x+2)}{x-2}=x+2, \quad x\neq 2.
\]

\medskip

\textbf{Example 4: Solving an Equation}

\textbf{Problem.}  
Solve
\[
2x-7=5.
\]

\textbf{Solution.}

\[
2x=12 \Rightarrow x=6.
\]

\subsection*{Practice Problems}

\begin{enumerate}
  \item Simplify the expression:
  \[
  4x-7+3x.
  \]
  \workbox[1.2in]{}

  \item Factor completely:
  \[
  x^2-16.
  \]
  \workbox[1.6in]{}

  \item Simplify the rational expression:
  \[
  \frac{x^2-5x}{x}, \quad x\neq 0.
  \]
  \workbox[1.8in]{}

  \item Solve the equation:
  \[
  3x+4=19.
  \]
  \workbox[1.6in]{}

  \item Explain why algebraic simplification is important before computing a limit.
  \workbox[2.2in]{}

  \begin{answer}
  \textbf{Algebraic Operations and Simplification, Practice Problems.}
  \begin{enumerate}
    \item $7x-7.$
    \item $(x-4)(x+4).$
    \item $x-5.$
    \item $x=5.$
    \item Simplification removes removable discontinuities and makes limits easier to evaluate.
  \end{enumerate}
  \end{answer}
\end{enumerate}

\subsection*{Section Summary}

\begin{itemize}
  \item Algebraic simplification is a foundational skill for calculus.
  \item Factoring allows cancellation in rational expressions.
  \item Rational expressions are undefined where the denominator is zero.
  \item Strong algebra skills reduce errors in limits, derivatives, and integrals.
\end{itemize}


\newpage

\section{Exponential and Logarithmic Functions}

\subsection*{Learning Objectives}
After completing this section, you should be able to:
\begin{itemize}
  \item Recognize exponential and logarithmic functions.
  \item Apply laws of exponents and logarithms.
  \item Evaluate exponential and logarithmic expressions.
  \item Solve basic exponential and logarithmic equations.
  \item Interpret exponential models in business contexts.
\end{itemize}

\subsection*{Key Definitions and Concepts}

\textbf{Exponential Function.}  
An exponential function has the form
\[
f(x)=ab^x,
\]
where $a\neq 0$ and $b>0$, $b\neq 1$.

Exponential functions are commonly used to model growth and decay in business, such as revenue growth, inflation, or population change.

\medskip

\textbf{Natural Exponential Function.}  
The function
\[
f(x)=e^x
\]
uses the base $e\approx 2.71828$ and appears frequently in calculus.

\medskip

\textbf{Logarithmic Function.}  
A logarithmic function is the inverse of an exponential function.  
The natural logarithm is written as
\[
f(x)=\ln x,
\]
and is defined for $x>0$.

\medskip

\textbf{Laws of Logarithms.}
\[
\ln(ab)=\ln a+\ln b,\qquad
\ln\!\left(\frac{a}{b}\right)=\ln a-\ln b,\qquad
\ln(a^r)=r\ln a.
\]

\subsection*{Solved Examples}

\textbf{Example 1: Evaluating an Exponential Expression}

\textbf{Problem.}  
Evaluate
\[
f(2)=3e^{2}.
\]

\textbf{Solution.}

\[
f(2)=3e^2.
\]

\medskip

\textbf{Example 2: Simplifying Logarithms}

\textbf{Problem.}  
Simplify
\[
\ln(5x^2).
\]

\textbf{Solution.}

Using logarithm laws,
\[
\ln(5x^2)=\ln 5+2\ln x.
\]

\medskip

\textbf{Example 3: Solving an Exponential Equation}

\textbf{Problem.}  
Solve
\[
e^{2x}=7.
\]

\textbf{Solution.}

Take the natural logarithm of both sides:
\[
2x=\ln 7 \Rightarrow x=\frac{1}{2}\ln 7.
\]

\medskip

\textbf{Example 4: Business Interpretation}

\textbf{Problem.}  
A company’s revenue is modeled by
\[
R(t)=500e^{0.04t},
\]
where $t$ is measured in years. Interpret the rate $0.04$.

\textbf{Solution.}

The revenue grows continuously at a rate of 4\% per year.

\subsection*{Practice Problems}

\begin{enumerate}
  \item Evaluate the expression:
  \[
  2e^{3}.
  \]
  \workbox[1.2in]{}

  \item Simplify the logarithmic expression:
  \[
  \ln(4x).
  \]
  \workbox[1.6in]{}

  \item Solve the equation:
  \[
  e^{x}=10.
  \]
  \workbox[1.6in]{}

  \item Solve the equation:
  \[
  \ln x=3.
  \]
  \workbox[1.6in]{}

  \item A population grows according to
  \[
  P(t)=1000e^{0.02t}.
  \]
  Explain the meaning of the growth rate.
  \workbox[2.2in]{}

  \begin{answer}
  \textbf{Exponential and Logarithmic Functions, Practice Problems.}
  \begin{enumerate}
    \item $2e^3.$
    \item $\ln 4+\ln x.$
    \item $x=\ln 10.$
    \item $x=e^3.$
    \item The population grows continuously at a rate of 2\% per year.
  \end{enumerate}
  \end{answer}
\end{enumerate}

\subsection*{Section Summary}

\begin{itemize}
  \item Exponential functions model growth and decay.
  \item Logarithms are inverses of exponential functions.
  \item Logarithm laws simplify expressions and equations.
  \item Exponential models are fundamental in business and economics.
\end{itemize}
