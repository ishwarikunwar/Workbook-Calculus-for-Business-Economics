
\chapter{Selected Answers to Practice Problems}
\section*{Review}
\subsection*{Functions and Graphs}

\begin{enumerate}
  \item
  \begin{enumerate}
    \item $f(5)=2(5)+1=11.$
    \item $2x+1=9 \Rightarrow 2x=8 \Rightarrow x=4.$
  \end{enumerate}

  \item The domain is all real numbers except $x=-3$.

  \item
  \begin{enumerate}
    \item $R(0)$ is the revenue when no units are sold (typically \$0).
    \item A decreasing $R(q)$ means revenue goes down as the number of units sold increases.
  \end{enumerate}

  \item The graph of $y=-3$ is a horizontal line 3 units below the $x$-axis.
\end{enumerate}

\subsection*{Algebraic Operations and Simplification}

\begin{enumerate}
  \item $4x-7+3x=7x-7.$

  \item $x^2-16=(x-4)(x+4).$

  \item $\dfrac{x^2-5x}{x}=\dfrac{x(x-5)}{x}=x-5,\quad x\neq 0.$

  \item $3x+4=19 \Rightarrow 3x=15 \Rightarrow x=5.$

  \item Simplification helps remove removable discontinuities (such as common factors that cancel), making limits easier to evaluate and reducing algebra errors.
\end{enumerate}

\subsection*{Exponential and Logarithmic Functions}

\begin{enumerate}
  \item $2e^3.$

  \item $\ln(4x)=\ln 4+\ln x.$

  \item $e^x=10 \Rightarrow x=\ln 10.$

  \item $\ln x=3 \Rightarrow x=e^3.$

  \item The growth rate $0.02$ means the population grows continuously at about 2\% per year.
\end{enumerate}


\section*{The Derivative}
\subsection*{Limits and Continuity}

\begin{enumerate}
  \item
  \begin{enumerate}
    \item $380$
    \item Revenue approaches \$380 as output approaches 25 units.
  \end{enumerate}

  \item
  \begin{enumerate}
    \item $500$
    \item $500$
    \item Continuous at $x=40$.
  \end{enumerate}

  \item
  A limit depends on values near the point, not the value at the point.
\end{enumerate}

\subsection*{The Derivative}

\begin{enumerate}
  \item
  \begin{enumerate}
    \item $C'(x) = 0.02x + 4$.
    \item $C'(80) = 0.02(80) + 4 = 5.6$.
    \item At 80 units, cost is increasing at about \$5.60 per additional unit (marginal cost).
  \end{enumerate}

  \item
  \begin{enumerate}
    \item $R'(x) = 150 - 0.5x$.
    \item $R'(60) = 150 - 30 = 120$.
    \item At 60 units, revenue is increasing at about \$120 per additional unit (marginal revenue).
  \end{enumerate}

  \item
  \begin{enumerate}
    \item Left slope: $(27 - 22)/1 = 5$.  
          Right slope: $(31 - 27)/1 = 4$.  
          Average: $(5 + 4)/2 = 4.5$.
    \item At week 4, profit is increasing at about \$4.5 thousand per week (about \$4{,}500 per week).
  \end{enumerate}
\end{enumerate}

\subsection*{Power and Sum Rules for Derivatives}

\begin{enumerate}
  \item $f'(x)=30x^5$.

  \item $g'(x)=12x^2-14x$.

  \item
  \begin{enumerate}
    \item $C'(x)=0.06x^2+12x+150$.
    \item $C'(40)=726$.
  \end{enumerate}

  \item
  \begin{enumerate}
    \item $R'(x)=120-0.6x$.
    \item At 50 units, revenue is increasing at about \$90 per additional unit.
  \end{enumerate}
\end{enumerate}

\subsection*{Product and Quotient Rules}

\begin{enumerate}
  \item $f'(x)=4(x^2+3)+4x(2x)=12x^2+12$.

  \item $g(x)=\dfrac{6x^2+50x}{x}=6x+50$, so  $g'(x)=6.$
 

  \item
  \begin{enumerate}
    \item $R'(x)=120-6x$.
    \item $R'(10)=60$; at 10 units sold, revenue is increasing at about \$60 per additional unit.
  \end{enumerate}

  \item
  \begin{enumerate}
    \item $A'(x)=0.05-\dfrac{2000}{x^2}$.
    \item At $x=50$, $A'(50)=0.05-\dfrac{2000}{2500}=-0.75$, so average cost is decreasing.
  \end{enumerate}
\end{enumerate}

\subsection*{Chain Rule}

\begin{enumerate}
  \item $f'(x)=5(x^2+3x+1)^4(2x+3)$.

  \item $g'(x)=\dfrac{2x+5}{\sqrt{2x^2+10x}}$.

  \item
  \begin{enumerate}
    \item $R'(x)=-6(100-2x)^2$.
    \item $R'(20)=-6(60)^2=-21{,}600$; revenue is decreasing at about \$21{,}600 per additional unit.
  \end{enumerate}

  \item
  \begin{enumerate}
    \item $P'(x)=\dfrac{x+5}{\sqrt{x^2+10x+25}}$.
    \item $P'(25)=\dfrac{30}{\sqrt{900}}=1$.
  \end{enumerate}
\end{enumerate}


\subsection*{Second Derivative and Concavity}

\begin{enumerate}
  \item $f'(x)=4x^3-24x^2+36x$, \quad $f''(x)=12x^2-48x+36$.

  \item
  \begin{enumerate}
    \item $R'(x)=9x^2-60x+90$, \quad $R''(x)=18x-60$.
    \item $R''(x)=0$ at $x=\dfrac{10}{3}$.
    Concave down for $x<\dfrac{10}{3}$ and concave up for $x>\dfrac{10}{3}$.
  \end{enumerate}

  \item
  \begin{enumerate}
    \item $C''(x)=1.5x-12$.
    \item $C''(x)=0$ at $x=8$, so there is a point of inflection at $x=8$.
  \end{enumerate}
\end{enumerate}

\subsection*{Optimization}

\begin{enumerate}
  \item
  \begin{enumerate}
    \item $P'(x)=-2x+120=0 \Rightarrow x=60$.
    \item $P(60)=2700$.
  \end{enumerate}

  \item
  \begin{enumerate}
    \item $A'(x)=1-\dfrac{1000}{x^2}=0 \Rightarrow x=\sqrt{1000}\approx31.6$.
    \item Average cost is minimized at about 32 units.
  \end{enumerate}

  \item
  \begin{enumerate}
    \item $R'(x)=-6x+180=0 \Rightarrow x=30$.
    \item $R(30)=2700$.
  \end{enumerate}

  \item
  \begin{enumerate}
    \item $P(x)=80x-(x^2+40x+600)=-x^2+40x-600$.
    \item $P'(x)=-2x+40=0 \Rightarrow x=20$.
  \end{enumerate}
\end{enumerate}

\subsection*{Curve Sketching}

\begin{enumerate}
  \item
  \begin{enumerate}
    \item $f'(x)=3x^2-18x+24=3(x-2)(x-4)$, so the critical points are $x=2$ and $x=4$.
    \item Increasing on $(2,4)$; decreasing on $(-\infty,2)$ and $(4,\infty)$.
    \item $f''(x)=6x-18$, so $f''(x)=0$ at $x=3$ (inflection point at $x=3$).
  \end{enumerate}
  \item
  \begin{enumerate}
    \item $R'(x)=-3x^2+30x-50=0 \Rightarrow x=5\pm \dfrac{5\sqrt{3}}{3}$.
    \item Revenue is increasing on $\left(5-\dfrac{5\sqrt{3}}{3},\,5+\dfrac{5\sqrt{3}}{3}\right)$ and decreasing outside this interval.
    \item $R''(x)=-6x+30$. Concave up for $x<5$ and concave down for $x>5$.
  \end{enumerate}

  \item
  \begin{enumerate}
    \item $C'(x)=4x^3-12x^2+12x=4x(x-1)(x-3)$, \quad
          $C''(x)=12x^2-24x+12=12(x-1)^2$.
    \item Since $C''(x)=12(x-1)^2\ge 0$ for all $x$, $C$ is concave up for all $x$ (no inflection point).
  \end{enumerate}
\end{enumerate}
\subsection*{Applied Optimization}

\begin{enumerate}
  \item
  \begin{enumerate}
    \item Let $x$ be units of Product X. Then $120-x$ units of Product Y:
    \[
    P(x)=40x+25(120-x)=15x+3000.
    \]
    \item Since $P(x)$ is increasing, profit is maximized at $x=120$.
    Produce 120 units of Product X and 0 units of Product Y.
  \end{enumerate}

  \item
  \begin{enumerate}
    \item Let $x$ be the width and $y$ the length:
    \[
    2x+y=200 \Rightarrow y=200-2x,
    \quad A(x)=xy=x(200-2x).
    \]
    \item
    \[
    A'(x)=200-4x=0 \Rightarrow x=50,\quad y=100.
    \]
  \end{enumerate}

  \item
  \begin{enumerate}
    \item Revenue as a function of price:
    \[
    R(p)=pq=p(500-5p)=500p-5p^2.
    \]
    \item
    \[
    R'(p)=500-10p=0 \Rightarrow p=50.
    \]
  \end{enumerate}
\end{enumerate}

 
\subsection*{Implicit Differentiation and Related Rates}

\begin{enumerate}
  \item Differentiate:
  \[
  x^2+y^2+4xy=16
  \]
  \[
  2x+2y\frac{dy}{dx}+4\left(x\frac{dy}{dx}+y\right)=0
  \]
  \[
  (2y+4x)\frac{dy}{dx}=-(2x+4y)
  \quad\Rightarrow\quad
  \frac{dy}{dx}=-\frac{x+2y}{y+2x}.
  \]

  \item $A=lw \Rightarrow \dfrac{dA}{dt}=l\dfrac{dw}{dt}+w\dfrac{dl}{dt}$.
  At $l=20$, $w=10$, $\dfrac{dl}{dt}=3$, $\dfrac{dw}{dt}=-1$:
  \[
  \frac{dA}{dt}=20(-1)+10(3)=10 \text{ m}^2/\text{month}.
  \]

  \item $V=pq \Rightarrow \dfrac{dV}{dt}=p\dfrac{dq}{dt}+q\dfrac{dp}{dt}$.
  At $p=25$, $q=400$, $\dfrac{dp}{dt}=1$, $\dfrac{dq}{dt}=-30$:
  \[
  \frac{dV}{dt}=25(-30)+400(1)=-350 \text{ dollars/month}.
  \]
\end{enumerate}

\section*{The Integral}
\subsection*{The Definite Integral}

\begin{enumerate}
  \item
  \begin{enumerate}
    \item The integral represents the total change in profit as production increases from 10 units to 40 units.
    \item A negative value indicates that overall profit decreases (a net loss) over that production range.
  \end{enumerate}

  \item
  \begin{enumerate}
    \item The integral represents the total change in cost as production increases from $x=a$ to $x=b$.
    \item It gives the accumulated cost incurred over that interval of production.
  \end{enumerate}

  \item The integral represents the net change in inventory over the 8-week period.
\end{enumerate}

\subsection*{The Fundamental Theorem of Calculus and Antidifferentiation}

\begin{enumerate}
  \item
  
  $\int_{1}^{4} (2x^2-x)\,dx
  =
  \left[\frac{2}{3}x^3-\frac{1}{2}x^2\right]_{1}^{4}
  =
  \left(\frac{128}{3}-8\right)-\left(\frac{2}{3}-\frac{1}{2}\right)
  =30.$
  

  \item
    $\int_{5}^{20} (6x+10)\,dx
  =
  \left[3x^2+10x\right]_{5}^{20}
  =
  (1200+200)-(75+50)
  =1725.$
 

  \item
 
  $\int_{10}^{30} (50-2x)\,dx
  =
  \left[50x-x^2\right]_{10}^{30}
  =
  (1500-900)-(500-100)
  =600.$
 
\end{enumerate}


\subsection*{Antiderivatives of Formulas}

\begin{enumerate}
  \item $\int (6x^2-4x+9)\,dx = 2x^3-2x^2+9x+C.$

  \item $\int (2e^x-3\cdot4^x)\,dx = 2e^x-\dfrac{3\cdot4^x}{\ln 4}+C.$

  \item $\int \dfrac{5}{x}\,dx = 5\ln|x|+C.$

  \item $\int (x^{-3}+2x^{1/2})\,dx
  = -\dfrac{1}{2}x^{-2}+\dfrac{4}{3}x^{3/2}+C.$
\end{enumerate}


\subsection*{Substitution}

\begin{enumerate}
  \item $\int x(2x^2+5)^3\,dx=\dfrac{(2x^2+5)^4}{8}+C.$

  \item $\int e^{4x+1}\,dx=\dfrac{1}{4}e^{4x+1}+C.$

  \item $\int \dfrac{6x}{x^2+4}\,dx=3\ln(x^2+4)+C.$
\end{enumerate}

\subsection*{Additional Integration Techniques}

\begin{enumerate}
  \item $\int x\ln x\,dx=\dfrac{x^2}{2}\ln x-\dfrac{x^2}{4}+C.$

  \item $\int_1^e \ln x\,dx=1.$

  \item $\int \dfrac{1}{x^2-16}\,dx=\dfrac{1}{8}\ln\left|\dfrac{x-4}{x+4}\right|+C.$
\end{enumerate}

\subsection*{Area, Volume, and Average Value}

\begin{enumerate}
  \item $\int_0^3 4x\,dx=18.$

  \item $\int_0^2 (2x-x^2)\,dx=\dfrac{4}{3}.$

  \item $V=\int_0^1 \pi(x^2)^2\,dx=\pi\int_0^1 x^4\,dx=\dfrac{\pi}{5}.$

  \item $f_{\text{avg}}=\dfrac{1}{3-1}\int_1^3 (x^2+2)\,dx=\dfrac{13}{3}.$
\end{enumerate}

\subsection*{Applications to Business}

\begin{enumerate}
  \item
  \begin{enumerate}
    \item $q^*=25,\; p^*=35.$
    \item $\text{CS}=\int_0^{25}(60-q)\,dq-35(25)=\dfrac{625}{2}.$
    \item $\text{PS}=35(25)-\int_0^{25}(q+10)\,dq=\dfrac{625}{2}.$
  \end{enumerate}

  \item
  \begin{enumerate}
    \item $\text{PV}=\int_0^{5}1000e^{-0.06t}\,dt=\dfrac{1000}{0.06}\left(1-e^{-0.3}\right).$
    \item $\text{FV}=\text{PV}\,e^{0.06\cdot 5}=\dfrac{1000}{0.06}\left(e^{0.3}-1\right).$
  \end{enumerate}
\end{enumerate}


\subsection*{Differential Equations}

\begin{enumerate}
  \item $B'(t)=0.03B(t)-15.$

  \item Not a solution.

  \item $y^2=4x^2+C.$

  \item $\dfrac{dy}{dt}=0.05y,\quad y=Ae^{0.05t}.$

  \item $\dfrac{dy}{dt}=0.02(10000-y).$
\end{enumerate}

\section*{Functions of Two Variables}
\subsection*{Functions of Two Variables}

\begin{enumerate}
  \item $R(q,p)$ represents the revenue generated by selling $q$ units at price $p$.

  \item
  \begin{enumerate}
    \item $C(4,10)=5(4)+2(10)=40.$
    \item The total cost for $x=4$ and $y=10$ is \$40.
  \end{enumerate}

  \item $C(2,5)=3(2)+4(5)=26$ and $C(5,2)=3(5)+4(2)=23$; they are not equal because the order of inputs changes the result.

  \item The graph of $z=-1$ is a plane parallel to the $xy$-plane, 1 unit below it.

  \item $\sqrt{(5-2)^2+(4-0)^2+(5-1)^2}=\sqrt{34}.$
\end{enumerate}

\subsection*{Calculus of Functions of Two Variables}

\begin{enumerate}
  \item
  \begin{enumerate}
    \item $f_x=2x,\quad f_y=6y.$
    \item $f_x(2,1)=4,\quad f_y(2,1)=6.$
  \end{enumerate}

  \item $C_y(x,y)$ measures the rate at which cost changes as materials $y$ change, holding labor $x$ constant.

  \item $g_x=\dfrac{1}{3}e^{x+y},\quad g_y=\dfrac{1}{3}e^{x+y}+\dfrac{1}{y}.$

  \item $f(2.1,2.9)\approx f(2,3)+1.2(0.1)+(-0.5)(-0.1)=f(2,3)+0.07.$

  \item Partial derivatives are useful because they approximate how the output changes for small changes in one input while the other input is held fixed, enabling quick estimation and sensitivity analysis.
\end{enumerate}

\subsection*{Optimization}

\begin{enumerate}
  \item Critical point $(1,-2)$.
  
  $f_{xx}=2,\; f_{yy}=2,\; f_{xy}=0$ gives  
  $D=2\cdot 2-0=4>0$ and $f_{xx}>0$, so a local minimum.

  \item Critical point $(0,0)$.
  
  $f_{xx}=2,\; f_{yy}=-2,\; f_{xy}=4$ gives  
  $D=2(-2)-4^2=-16<0$, so a saddle point.

  \item The profit function is maximized at $(x,y)=(12,7)$.
\end{enumerate}

